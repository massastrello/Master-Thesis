%\the\textwidth = 379.37pt
\chapter{Hybrid Port--Hamiltonian Systems}

\label{chap:HPH_systems}
\minitoc

\thispagestyle{empty}

\newpage
%%%%%%%%%%%%%%%%%%%%%%%%%%%%%%%%%%%%%%%%%%%%%%%%%%%%%%%%%%%%%%%%%%%%%%%%%%%%%%%
\section{Introduction}
\lettrine[lines=4]{\color{brickred}H}{ereafter}, the framework of hybrid port--Hamiltonian systems will be introduced. Starting from the results discussed in Chapter \ref{chap:preliminaries}, the theories of port--Hamiltonian systems and hybrid inclusions will be merged. 
%
\newline

%
First, the \textit{impulsive port--Hamiltonian systems} will be characterized. It is a generalization of single--flowed hybrid inclusions in a port--Hamiltonian fashion. This class of system is relevant to this study as most mechanical systems exhibiting (partially) elastic impacts, including robotics applications, belong to this framework. 
%
\newline

%
Then, the general \textit{hybrid port--Hamiltonian system} will be defined. Examples for both classes of proposed models will be provided.
%
\newline

%
Then, the concept of passivity will be extended to the hybrid case in the port--Hamiltonian context. Finally Lyapunov stability theorem is also extended from the one of hybrid inclusions. 
%
\clearpage
%%%%%%%%%%%%%%%%%%%%%%%%%%%%%%%%%%%%%%%%%%%%%%%%%%%%%%%%%%%%%%%%%%%%%%%%%%%%%%%
\section{Definitions and Basic Assumptions}
%%%%%%%%%%%%%%%%%%%%%%%%%%%%%%%%%%%%%%%%%%%%%%%% 
\subsection{Impulsive Port--Hamiltonian Systems}
%
Consider the single--flow specialization of the hybrid inclusion (\ref{eq:hs}):
%
\begin{equation}\label{eq:hs_singleflow}
    %
    \left\{ 
        \begin{matrix*}[l]\vspace{1pt}
            %
            \dot{\xb} = \mathbf{f}(\xb,\ub) && (\xb,\ub)\in\C\times\U\\
            \xb^+ \in \G(\xb)&&\quad~~\xb\in\D\\
            \yb = \mathbf{h}(\xb)
            %
        \end{matrix*}
    \right.
    %
\end{equation}
%    
with $\mathbf{f}:\R^n\times\R^m\rightarrow\R^n$.
Here it is assumed that inputs enter into the dynamics only during flows and not during jumps. Therefore, 
%
\begin{equation}
    \ub\triangleq\ub_c,~~\U\triangleq\U_c
\end{equation}
%
This assumption has been made to simplify the model since in most physical systems, no impulsive forcing terms can be practically applied.
%
\begin{defn}[Impulsive port--Hamiltonian Systems]
%
An impulsive port--Hamiltonian system is a system in the form (\ref{eq:hs_singleflow})  with port--Hamiltonian flows and output, i.e.
%
\begin{equation}\label{eq:impulsive_ph}
    %
    \boxed{
    \left\{ 
        \begin{matrix*}[l]\vspace{1pt}
            %
            \dot{\xb} = \left[\Jb(\xb)-\Rb(\xb)\right]\dH(\xb) + \Gb(\xb)\ub && (\xb,\ub)\in\C\times\U\\
            \xb^+ \in \G(\xb)&&\quad~~ \xb\in\D\\
            \yb = \Gb^\top(\xb)\dH(\xb)
            %
        \end{matrix*}
    \right.
    }
    %
\end{equation}
%
The state--space of the system is $\X\triangleq\C\cup\D$ and $\Ha:\X\rightarrow\R$.
\end{defn}
%
An example of this type of systems is the \textit{ball--dribbling} and \textit{ball--juggling} robots of Fig. \ref{fig:1D}. Note that most mechanical systems exhibiting impacts admit a representation (\ref{eq:impulsive_ph}).
%
\begin{exmp}[Impact Pendulum]\label{ex:ipend}
    Consider an impact pendulum of mass $m$ and length $\ell$ as in Fig. \ref{fig:ipend}.
    %
    \begin{figure}
        \centering
        \begin{tikzpicture}[thick,>=latex,->]

\begin{scope}
\clip(-5,2) rectangle (5,-5);

\draw[dashed] (0,0)  circle (4.24cm);
\filldraw[white] (-4.3,4.3) rectangle (4.3,0);
\filldraw[white] (-4.3,4.3) rectangle (0,-4.3);


\draw[double distance=1.6mm] (0,0) -- (3,-3) node[midway,xshift=4mm,yshift=2mm]{$\ell$};
\draw[fill=white] (-1.2,1.0) -- (-.5,0) arc(180:360:0.5) -- (1.2,1.0) -- cycle;
\draw[draw=black,fill=white] (0, 0) circle circle (.3cm);
\draw[draw=black,fill=gray!50] (3,-3) circle circle (.3cm);
\draw[-,dotted] (0,0) -- (0,-1);

\draw[pattern=north east lines] (-1.4,1.3) rectangle (1.4,1);
\node at (.3,-.8) {$q$};   

\draw[-] (3,-3) node[] {$m$};
\draw[pattern=north east lines] (-.6,-4.74) rectangle (-.3,-3.84);
\draw[pattern=north east lines] (3.84,.3) rectangle (4.74,.6);
\draw[-] (-1,-4.24) node[] {$c_l$};
\draw[-] (4.24,1) node[] {$c_u$};
\end{scope}

\end{tikzpicture}
        \caption[Impact pendulum]{Impact pendulum of mass $m$ and length $\ell$. The systems has two impacts, one when the pendulum is at the resting position ($q=0$) and at $q = \pi/2$.}
        \label{fig:ipend}
    \end{figure}
    %
    Let $q$ be the pendulum angle and $p \triangleq m \ell^2 \dot{q}$ its angular momentum. The flows of the system are:
    %
    \begin{equation}
        %
        \dot{q} = \frac{1}{m\ell^2}p \quad
        \dot{p} = -m\gamma\ell\sin(q)-\frac{\beta}{m\ell^2}p + u
        %
    \end{equation}
    %
    where $u$ is the input torque applied at the joint, $\gamma$ is the gravitational constant and $\beta$ is the viscous friction coefficient in the joint. The system admits an impulsive port--Hamiltonian form (\ref{eq:impulsive_ph}) with state $\xb\triangleq(q,p)$. In particular, the flows are described by the system matrices:
    %
    \begin{equation}
        \Jb\triangleq\begin{bmatrix}0&1\\-1&0\end{bmatrix},~~\Rb\triangleq \begin{bmatrix}0&0\\0&\beta\end{bmatrix},~~\Gb\triangleq\begin{bmatrix}0\\1\end{bmatrix}.
    \end{equation}
    %
    The Hamiltonian function is 
    %
    \begin{equation}
        \Ha(q,p)\triangleq\frac{1}{2m\ell^2}p^2 + mg\ell(1-\cos(q))
    \end{equation}
    %
    The systems has two impacts, one when the pendulum is at the resting position ($q=0$) and at $q = \pi/2$. Thus,
    %
    \begin{align}
        &\xb^+ = \mathbf{g}_1(\xb) \triangleq \overbrace{\begin{bmatrix}1&0\\0&-c_l\end{bmatrix}}^{\mathbf{M}_1}\xb\quad\text{if}\quad\xb\in\D_1\triangleq\left\{\xb:q=0\land p\leq 0\right\} \\
        %
        &\xb^+ = \mathbf{g}_2(\xb) \triangleq \underbrace{\begin{bmatrix}1&0\\0&-c_u\end{bmatrix}}_{\mathbf{M}_2}\xb\quad\text{if}\quad\xb\in\D_2\triangleq\left\{\xb:q=\pi/2\land p\geq 0\right\}
    \end{align}
    %
    and, therefore,
    %
    \begin{equation}
        \G\triangleq\left\{\mathbf{g}_i:\xb\in\D_i\Rightarrow\xb^+ = \mathbf{g}_i(\xb),~i = 1,2\right\}
    \end{equation}
    %
    \end{exmp}
%
\subsection{Hybrid port--Hamiltonian Systems}
%

In this thesis, the type of systems which are dealt with present a finite number of modes and that can be thus described through an \textit{hybrid automata} (see \cite{van2000introduction}). However, it is convenient to describe them as hybrid inclusions in the form (\ref{eq:hs}). In \citep{goebel2009hybrid} an hybrid inclusion formulation of hybrdid automata is given. However, the formulation of hybrid port--Hamiltonian systems will be derived starting from the notion of impulsive port--Hamiltonian system.

Let us consider a dynamical systems which behavior is the collection of $r$ ``modes'' in the form (\ref{eq:impulsive_ph}), i.e.
%
\begin{equation}
    %
    \left\{ 
        \begin{matrix*}[l]\vspace{1pt}
            %
            \dot{\xb} = \left[\Jb_s(\xb)-\Rb_s(\xb)\right]\dH_s(\xb) + \Gb_s(\xb)\ub && (\xb,\ub)\in\C_s\times\U\\
            \xb^+ \in \G_s(\xb)&&\quad~~ \xb\in\D_s\\
            \yb = \Gb_s^\top(\xb)\dH_(\xb)
            %
        \end{matrix*}\qquad s\in\M 
    \right.
    %
\end{equation}
%
where $\M\triangleq\mathbb{N}^*_{\leq r}$. The transition between one mode and another may happen only during jumps. Therefore, a state extension might be performed to include the mode index in the state:
%
\begin{equation}
    \zb \triangleq (\xb,s).
\end{equation}
%
Let us now define the port--Hamiltonian flow set--valued mapping as
%
\begin{equation}\label{eq:hph_flowmap}
    \F_{\tt PH}(\xb,\ub) \triangleq \F(\xb,\ub)\times\{0\}
\end{equation}
%
where
%
\begin{equation}
    \F\triangleq\left\{\mathbf{f}_i(\xb,\ub) = \left[\Jb_i(\xb)-\Rb_i(\xb)\right]\dH_i(\xb) + \Gb_i(\xb)\ub~:~(\xb,s)\in\C_i\times\{i\}\Rightarrow \dot{\xb} = \mathbf{f}_i(\xb,\ub)\right\}.
\end{equation}
%
The flow set can be then defined as
%
\begin{equation}
    \C\triangleq\bigcup_{s\in\M}\left(\C_s\times\{s\}\right).
\end{equation}
%
Note that the jump set $\D_s$ and the jump map $\G_s(\xb)$ of each mode enjoy the following:
%
\begin{equation}
    \D_s\triangleq \bigcup_{i\in\M}\D_{s\Arrow{0.1cm} i}~~\text{ and }~~\G_s(\xb)\triangleq \bigcup_{i\in\M}\mathbf{g}_{s\Arrow{0.1cm}i}(\xb)
\end{equation}
%
such that 
%
\begin{equation}
    %
    \begin{bmatrix}
        \xb^+\\
        s^+    
    \end{bmatrix}
    %
    =
    %
    \begin{bmatrix}
        \mathbf{g}_{s\Arrow{0.1cm}i}(\xb)\\
        i    
    \end{bmatrix}
    %
    \qquad (\xb,s)\in\D_{s\Arrow{0.1cm}i}\times\{s\}.
\end{equation}
%
The port--Hamiltonian jump set--valued mapping is then defined as
%
\begin{equation}\label{eq:hph_jumpmap}
    \G_{\tt PH}(\x) \triangleq\bigcup_{s\in\M,~i\in\M}\left(\mathbf{g}_{s\Arrow{0.1cm}i}(\xb),~i\right),
\end{equation}
%
while the jump set as
%
\begin{equation}\label{eq:hph_jumpset}
    \D \triangleq \bigcup_{s\in\M}\left(\D_s\times\{s\}\right).
\end{equation}
%
If the output set valued mapping is also defined as
%
\begin{equation}
    \Oo_{\tt PH} \triangleq \left\{\mathbf{h}_i(\xb) = \mathbf{G}_i^\top(\xb)\dH(\xb)~:~(\xb,s) \in\C_s\times\{i\}\Rightarrow y=\mathbf{h}_i(\xb)\right\},
\end{equation}
%
the formulation of \textit{hybrid port--Hamiltonian system} is the following:
%
\begin{equation}\label{eq:hph}
    \boxed{
    \left\{
        %
        \begin{matrix*}[l]
            %
            (\dot{\xb},\dot{s}) \in \F_{\tt PH} & (\xb,s)\in\C\\
            %
            (\xb^+,s^+) \in \G_{\tt PH} & (\xb,s)\in\D\\
            %
            y \in \Oo_{\tt PH}
            %
        \end{matrix*}
        %
    \right.
    }
\end{equation}
%
\begin{rem}
	The output of both impulsive and  hybrid port--Hamiltonian system is defined only during flows for two main reasons. Firstly, jumps occurrences are assumed to happen in zero--measure time intervals. Moreover, it is also assumed that no inputs are applied during jumps and, hence, undefined outputs cannot broke duality.
\end{rem}
%
\begin{exmp}[Hopping Robot on elastic ground.]\label{ex:hopping}
    %
    Consider the hopping robot on elastic ground represented in Fig. \ref{fig:hopping} and described by \cite{Ishikawa2003}.
    %
    \begin{figure}[h]
        \centering
        \definecolor{ocean}{rgb}{0.00000,0.44700,0.74100}
%
\begin{tikzpicture}[scale=1, every node/.style={scale=1}]
        \tikzstyle{spring}=[thick,decorate,decoration={zigzag,pre length=0.3cm,post length=0.3cm,segment length=6}]
        \tikzstyle{damper}=[thick,decoration={markings,  
            mark connection node=dmp,
            mark=at position 0.5 with 
            {
                \node (dmp) [thick,inner sep=0pt,transform shape,rotate=-90,minimum width=15pt,minimum height=3pt,draw=none] {};
                \draw [thick] ($(dmp.north east)+(2pt,0)$) -- (dmp.south east) -- (dmp.south west) -- ($(dmp.north west)+(2pt,0)$);
                \draw [thick] ($(dmp.north)+(0,-5pt)$) -- ($(dmp.north)+(0,5pt)$);
            }
        }, decorate]
        \tikzstyle{ground}=[fill,pattern=north east lines,draw=none,minimum width=0.75cm,minimum height=0.3cm,inner sep=0pt,outer sep=0pt]

        \node [draw, outer sep=0pt,thick,fill = gray!50] (M) [minimum width=2cm, minimum height=.3cm] {$m_1$};
        %
        \node [draw, outer sep=0pt,thick,fill = gray!50] (M2) [minimum width=2cm, minimum height=.3cm, yshift =-2cm] {$m_2$};
        %
        \node [draw, outer sep=0pt, thick,fill = gray] (M3) [minimum width=2cm, minimum height=0.1cm, yshift =-3.5cm] {};
        
        \draw [damper] (M.200) -- ($(M2.north west)!(M.200)!(M2.north east)$);
        \draw [spring] (M.340) -- ($(M2.north west)!(M.340)!(M2.north east)$);
        \draw [-,thick] (M.270) -- ($(M2.north west)!(M.270)!(M2.north east)$);
        \draw[draw=black,fill=ocean!50,thick] (0,-1cm) circle (0.25cm);
        
        \node (b) at (M2.north) [xshift = -1.2cm, yshift = 0.75cm] {$b$};
        \node (k) at (M2.north) [xshift = 1.2cm, yshift = 0.75cm] {$k$};
        \node (u) at (M2.north) [xshift = 0cm, yshift = 0.75cm] {$u$};
        
        \node (ground2) [ground,anchor=north,yshift=-2.25cm,minimum width=4cm,xshift=0cm] at (M2.south) {};
        \draw (ground2.north east) -- (ground2.north west);
        \draw (ground2.south east) -- (ground2.south west);
        \draw (ground2.north east) -- (ground2.south east);
        \draw (ground2.north west) -- (ground2.south west);
        
        \draw [spring] (ground2.20) -- ($(M3.south west)!(ground2.20)!(M3.south east)$);
        \draw [damper] (ground2.160) -- ($(M3.south west)!(ground2.160)!(M3.south east)$);
        
        \node (b) at (M3.south) [xshift = -1cm, yshift = -0.5cm] {$b_g$};
        \node (k) at (M3.south) [xshift = 1cm, yshift = -0.5cm] {$k_g$};
        %
        \draw[->, very thick] (ground2.west) -- (-2,1);
        \draw[-, dotted] (M.south west) -- +(-1,0cm);
        \draw[-, dotted] (M2.south west) -- +(-1,0cm);
        \draw[-, dotted] (M3.north west) -- +(-1,0cm);
        \node (q1) at (M.south west) [xshift = -1.3cm] {$q_1$};
        \node (q2) at (M2.south west) [xshift = -1.3cm] {$q_2$};
        \node (q3) at (M3.north west) [xshift = -1.6cm] {$q_3\equiv 0$};
        %%%%%%%%%%%%%%%%%%%%%%%%%%%%%%%%%%%%%%%
        \node [draw, outer sep=0pt,thick,fill = gray!50] (M) [minimum width=2cm, minimum height=.3cm,xshift = 5cm,yshift =-1.2cm] {$m_1$};
        %
        \node [draw, outer sep=0pt,thick,fill = gray!50] (M2) [minimum width=2cm, minimum height=.3cm, xshift = 5cm, yshift =-3.2cm] {$m_2$};
        %
        \node [draw, outer sep=0pt, thick,fill = gray] (M3) [minimum width=2cm, minimum height=0.1cm, xshift = 5cm, yshift =-3.5cm] {};
        
        \draw [damper] (M.200) -- ($(M2.north west)!(M.200)!(M2.north east)$);
        \draw [spring] (M.340) -- ($(M2.north west)!(M.340)!(M2.north east)$);
        \draw [-,thick] (M.270) -- ($(M2.north west)!(M.270)!(M2.north east)$);
        \draw[draw=black,fill=ocean!50,thick] (5cm,-2.2cm) circle (0.25cm);
        
        \node (b) at (M2.north) [xshift = -1.2cm, yshift = 0.75cm] {$b$};
        \node (k) at (M2.north) [xshift = 1.2cm, yshift = 0.75cm] {$k$};
        \node (u) at (M2.north) [xshift = 0cm, yshift = 0.75cm] {$u$};
        
        \node (ground2) [ground,anchor=north,yshift=-1.05cm,minimum width=4cm,xshift=0cm] at (M2.south) {};
        \draw (ground2.north east) -- (ground2.north west);
        \draw (ground2.south east) -- (ground2.south west);
        \draw (ground2.north east) -- (ground2.south east);
        \draw (ground2.north west) -- (ground2.south west);
        
        \draw [spring] (ground2.20) -- ($(M3.south west)!(ground2.20)!(M3.south east)$);
        \draw [damper] (ground2.160) -- ($(M3.south west)!(ground2.160)!(M3.south east)$);
        
        \node (b) at (M3.south) [xshift = -1cm, yshift = -0.5cm] {$b_g$};
        \node (k) at (M3.south) [xshift = 1cm, yshift = -0.5cm] {$k_g$};
        
        \node (fly) at (0,0) [xshift = 0cm, yshift = 1cm] {\textbf{Non--Contact Mode}};
        \node (fly) at (5cm,0) [xshift = 0cm, yshift = 1cm] {\textbf{Contact Mode}};
        
\end{tikzpicture}
        \caption[Hopping robot on elastic ground]{Hopping robot on elastic ground. }
        \label{fig:hopping}
    \end{figure}
    %
    The \textit{robot} is made up of two masses interconnected with a spring of stiffness $k$ and resting length $\ell$, a dashpot with damping coefficient $b$ and a linear actuator which exerts an axial force $u$ (input to the system). 
    $q_1,~q_2$ are the absolute height of the masses.
    
    The plate where the robots ``hops'' is assumed to be massless and connected to the ground with a spring of stiffness $k_g$ and a dashpot of damping coefficient $b_g$.
    $q_3$ is the position of the plate and assumed to be zero at rest.
    
    Finally, impacts are considered completely inelastic and that takeoff from ground always happens at $q_2 = q_3 = 0$.
    
    Following \citep{Ishikawa2003}, the system has two modes ``Contact'' and ``Non--contact''. The flows in contact/non--contact modes are, respectively:
    %
    \begin{align}
        %
        \text{non--contact}\quad
        %
        &\left\{
            \begin{matrix*}[l]
            %
            m_1\ddot{q}_1 = -m_1\gamma -k\left(q_1-q_2-\ell\right) -b\left(\dot{q}_1-\dot{q}_2\right) + u\\
            %
            m_2\ddot{q}_2 = -m_2\gamma +k\left(q_1-q_2-\ell\right) +b\left(\dot{q}_1-\dot{q}_2\right) - u\\
            %
            \ddot{q}_3 = 0
            \end{matrix*}
        \right.\\
        %%
        \text{contact}\quad
        %
        &\left\{
            \begin{matrix*}[l]
            %
            m_1\ddot{q}_1 = -m_1\gamma -k\left(q_1-q_2-\ell\right) -b\left(\dot{q}_1-\dot{q}_2\right) + u\\
            %
            m_2\ddot{q}_2 = -m_2\gamma +k\left(q_1-q_2-\ell\right) +b\left(\dot{q}_1-\dot{q}_2\right) - k_gq_3-b_g\dot{q}_3 - u\\
            %
            \ddot{q}_3 = \dot{q}_2
            \end{matrix*}
        \right.
        %
    \end{align}
    %
    Let $s$ be the mode index and let $s=1$ in ``non--contact'' while $s=2$ in contact modes. Moreover, let $\qb \triangleq (q_1,q_2,q_3)$, $\pb\triangleq (m_1q_1,m_2q_2)$ and $\xb\triangleq(\qb,\pb)$. The flows of the two system modes can be then transformed in port--Hamiltonian form with the following system matrices:
    %
    \begin{equation}
        %
        \Jb_1\triangleq
        \begin{bmatrix}
            %
            0&0&0&1&0\\
            0&0&0&0&1\\
            0&0&0&0&0\\
            -1&0&0&0&0\\
            0&-1&0&0&0\\
            %
        \end{bmatrix},~~
        %
        \Rb_1\triangleq
        \begin{bmatrix}
            %
            0&0&0&0&0\\
            0&0&0&0&0\\
            0&0&0&0&0\\
            0&0&0&b&-b\\
            0&0&0&-b&b\\
            %
        \end{bmatrix},~~
        %
        \Gb_1\triangleq
        \begin{bmatrix}
            %
            0\\0\\0\\1\\-1
            %
        \end{bmatrix},
        %
    \end{equation}
    %%
    \begin{equation}
        %
        \Jb_2\triangleq
        \begin{bmatrix}
            %
            0&0&0&1&0\\
            0&0&0&0&1\\
            0&0&0&0&1\\
            -1&0&0&0&0\\
            0&-1&-1&0&0\\
            %
        \end{bmatrix},~~
        %
        \Rb_2\triangleq
        \begin{bmatrix}
            %
            0&0&0&0&0\\
            0&0&0&0&0\\
            0&0&0&0&0\\
            0&0&0&b&-b\\
            0&0&0&-b&b+b_g\\
            %
        \end{bmatrix},~~
        %
        \Gb_2\triangleq
        \begin{bmatrix}
            %
            0\\0\\0\\1\\-1
            %
        \end{bmatrix}
        %
    \end{equation}
    %
    and Hamiltonian functions
    %
    \begin{align}
        &\Ha_1(\xb) \frac{1}{2}\triangleq \pb^\top\begin{bmatrix}m_1&0\\0&m_2\end{bmatrix}^{-1}\pb + \frac{1}{2}k\left(q_1-q_2-\ell\right)^2 + \gamma[m_1,m_2]\qb \\
        %
        &\Ha_2(\xb) \frac{1}{2}\triangleq \pb^\top\begin{bmatrix}m_1&0\\0&m_2\end{bmatrix}^{-1}\pb + \frac{1}{2}k\left(q_1-q_2-\ell\right)^2 + \frac{1}{2}k_g q_3^2+ \gamma[m_1,m_2]\qb.
    \end{align}
    %
    The port--Hamiltonian flow set--valued mapping can be then defined as in (\ref{eq:hph_flowmap}). Transitions between the two modes happen as follows:
    %
    \begin{align}
        &\begin{bmatrix}\xb^+\\s^+\end{bmatrix} = \begin{bmatrix}\xb\\2\end{bmatrix}\quad\text{if}\quad(\xb,s)\in\D_1\times\{1\}\triangleq\{\xb:q_2\leq 0\}\times\{1\}\\
        %
        &\begin{bmatrix}\xb^+\\s^+\end{bmatrix} = \begin{bmatrix}\xb\\1\end{bmatrix}\quad\text{if}\quad(\xb,s)\in\D_2\times\{2\}\triangleq\{\xb:q_2> 0\}\times\{2\}.
    \end{align}
    Thus, the jump set and the jump set--valued mapping can be derives as in (\ref{eq:hph_jumpset}), (\ref{eq:hph_jumpmap}).
    
    Finally, it can be noticed that the output is the same for each mode:
    %
    \begin{equation}
        y = \Oo_{\tt PH}\triangleq\Gb_1^\top\dH_1(\xb) = \Gb_2^\top\dH_2(\xb) = \dot{q}_1 - \dot{q}_2  
    \end{equation}
    %
\end{exmp}
%
%
The concept of passivity, fundamental for the
theory port-Hamiltonian systems, can be extended to
the hybrid case in the following section.
%
\clearpage
%
%%%%%%%%%%%%%%%%%%%%%%%%%%%%%%%%%%%%%%%%%%%%%%%%%%%%%%%%%%%%%%%%%%%%%%%%%%%%%%%
\section{Passivity}
%
The concept and characterization of passivity for hybrid systems has been recently explored by \cite{naldi2013passivity}.
%
\begin{defn}[Passive impulsive port--Hamiltonian system]\label{def:impulsive_passivity}
    %
    A system in the form (\ref{eq:impulsive_ph}) is passive if and only if it satisfies the following dissipation inequality
    %
    \begin{equation}
        \forall\xb\in\D,~\mathbf{g}\in\G \quad \Ha(\mathbf{g}(\xb))-\Ha(\xb)\leq 0
    \end{equation}
    %
\end{defn}
%
Hereafter, necessary and sufficient condition for the passivity of a subclass systems of type (\ref{eq:impulsive_ph}), is given. First, let us start with a basic definition.
%
\begin{defn}[Evenly $\bm\psi$--diverging function]
	Let $\bm\psi:\R^n\supseteq\dom(\bm\psi)\rightarrow\R^n$. A map $\Ha:\R^n\supseteq\dom(\Ha)\rightarrow\R$ is evenly diverging with respect to $\bm\psi$ if
	%
	\begin{equation}
	    \forall\bm\xi\in\dom(\bm\psi)~~\Ha(\bm\psi(\bm\xi))-\Ha(\bm\xi)\leq 0 \Leftrightarrow \|\bm\psi(\bm\xi)\|_2-\|\bm\xi\|_2\leq {0}
	\end{equation}
	%
\end{defn}
%
\begin{prop}[Condition for passivity of impulsive port--Hamiltonian systems]\label{prop:pass_impulsive}
	%
	Without any loss of generality, let us assume $\mymathbb{0}$ to be a minimum of $\Ha(\xb)$:
	%
	\begin{equation}
	    \mymathbb{0}_n = \min_{\x\in\X}\Ha(\xb).
	\end{equation}
	%
	A system (\ref{eq:impulsive_ph}) such that $\Ha(\xb)$ is positive in $\X\setminus\{\mymathbb{0}_n\}$, $\Ha(\mymathbb{0}_n)=0$ and $\Ha$ is evenly diverging with respect to all $\mathbf{g}\in\G$, is passive if and only if 
	%
	\begin{equation}\label{eq:nl_pass}
		\|\mathbf{g}\|_\infty\leq 1\quad\forall \mathbf{g}\in\G
	\end{equation}
	%
\end{prop}
%
\begin{proof}
	If $\Ha(\xb)$ is evenly diverging with respect to all the jump maps, it holds
	%
	\begin{align}
	    \forall\mathbf{g}\in\G,~\xb\in\X\quad\Ha(\mathbf{g}(\xb))-\Ha(\xb)\leq 0 \Leftrightarrow \|\mathbf{g}(\xb)\|_2 \leq \|\xb\|_2.
	\end{align}
	%
	Then,
	%
	\begin{align*}
	\Ha(\mathbf{g}(\xb))-\Ha(\xb)\leq 0 \quad \Leftrightarrow \quad \sup\limits_{\xb\neq 0}\frac{\|\mathbf{g}(\xb)\|_2}{\|\xb\|_2} \triangleq \|\mathbf{g}\|_\infty\leq 1
	\end{align*}
    %
\end{proof}
%
If $\mathbf{g}$ is a linear function, i.e., $\mathbf{g}(\xb) = \mathbf{M}\xb$ ($\mathbf{M}\in\R^{n\times n}$), the condition (\ref{eq:nl_pass}) corresponds in requiring that the maximum among the norms of the eigenvalues of $\mathbf{M}$ is less or equal than one:
%
\begin{equation}
	\forall k \in\mathbb{N}^*_{\leq n} \qquad \max\limits_{k}\|\lambda_k\|_2\leq 1\quad\lambda_k:\mathbf{M}\mathbf{v}_k = \lambda_k\mathbf{v}_k
\end{equation}
%
Note that this Proposition can be naturally extended to the case in which $\Ha$ has a unique minimum in $\xb^*$ by a change of coordinates $\zb\triangleq \xb-\xb^*$ and requiring $\Ha \circ~\zb:\X\rightarrow\R$ to be evenly diverging with respect to all $\gb\in\G$.
%
\newline

%
Here, passivity corresponds to the property of no internal energy generation during jumps. In fact, during flows, the (continuous time) passivity is guaranteed by the port--Hamiltonian structure.
Note that Definition \ref{def:impulsive_passivity} is analogous to the one of \textit{flow-passivity} in \cite{naldi2013passivity} applied to single--flows hybrid inclusions with port--Hamiltonian flows. 
%
\begin{rem}
    Recalling, example \ref{ex:ipend}, it can be concluded that the impact pendulum model is passive for Proposition \ref{prop:pass_impulsive} if and only if the restitution coefficients $c_l$, $c_u$ are all less then one.
\end{rem}
%
Passivity can also be defined for hybrid port--Hamiltonian systems by extending Definition \ref{def:impulsive_passivity} as follows.
%
\begin{defn} [Passive hybrid port--Hamiltonian system]\label{def:hybrid_passivity}
    %
    A system in the form (\ref{eq:hph}) is passive if and only if it satisfies the following dissipation inequality
    %
    \begin{equation}
        \forall s\in\M,~\xb\in\D_s,~\mathbf{g}\in\G_s \quad \Ha_s(\mathbf{g}(\xb))-\Ha_s(\xb)\leq 0
    \end{equation}
    %
\end{defn}
%
Similarly, Proposition \ref{prop:pass_impulsive} can extended to hybrid case.
%
\begin{prop}[Condition for passivity of hybrid port--Hamiltonian systems]\label{prop:pass_hybrid}
    %
	Let us assume
	%
	\begin{equation}
	    \forall s\in\M\quad \mymathbb{0}_n = \min_{\x\in\X}\Ha_s(\xb).
	\end{equation}
	%
	A system (\ref{eq:hph}) such that 
	%
	\begin{equation}
	    \forall s\in\M~~\left(\xb\in\X\setminus\{\mymathbb{0}_n\}\Rightarrow\Ha_s(\xb)>0\right) \land\Ha_s(\mymathbb{0}_n)=0
	\end{equation}
	%
	and, for all $s$, $\Ha_s$ is evenly diverging with respect to all $\mathbf{g}_s\in\G_s$, is passive if and only if 
	%
	\begin{equation}%\label{eq:nl_pass}
		\forall s\in\M,~\mathbf{g}_s\in\G\quad\|\mathbf{g}_s\|_\infty\leq 1.
	\end{equation}
	%
\end{prop}
%
\begin{proof}
    The proof follows naturally the one of Proposition \ref{prop:pass_impulsive}.
\end{proof}
%
\begin{rem}
The hopping robot of Example \ref{ex:hopping} is passive since the jump maps are simply the identity and, thus, there is no energy variation during jumps (jump--lossless). 
\end{rem}
%
Lyapunov stability of both impulsive and hybrid port--Hamiltonian systems with be addressed in the next section.
%
\clearpage
%%%%%%%%%%%%%%%%%%%%%%%%%%%%%%%%%%%%%%%%%%%%%%%%%%%%%%%%%%%%%%%%%%%%%%%%%%%%%%%
\section{Lyapunov Stability of the Autonomous Systems}
%
The theorems for both Impulsive and hybrid port--Hamiltonian systems have been derived from Theorem \ref{thm:hybrid_Lyap}.
%
\begin{thm}[Lyapunov Stability of Impulsive port--Hamiltonian Systems]\label{thm:impulsive_lyap}
    %
    Consider an system (\ref{eq:impulsive_ph}) with $\ub = 0$	and a compact set $\A\subset\X$ satisfying $\G\left(\D\cap\A\right)\subset\A$. If
	%
	\begin{subequations}
		\begin{align}
		&\Ha(\xb)>0&&\forall x\in \left(\C\cup\D\right)\setminus\A\label{eq:stabhph_1}\\
		&\Ha(\gb(\xb)) - \Ha(\xb)\leq 0 &&\forall \xb\in\D\setminus\A,\forall \gb\in\G\label{eq:stabhph_3}
		\end{align}
	\end{subequations}
	%
	then the set $\A$ is stable.
	%
\end{thm}
%
\begin{proof}
    The proof is obtained directly from Theorem \ref{thm:hybrid_Lyap}. First, let us assume $\Ha$ to be the candidate Lyapunov function. Then, conditions (\ref{eq:stabh_1}) and (\ref{eq:stabh_3}) correspond, respectively to (\ref{eq:stabhph_1}), (\ref{eq:stabhph_3}). Finally, condition (\ref{eq:stabh_2}) is automatically satisfied due to the port--Hamiltonian structure of the flows, i.e.
    %
    \begin{equation}
        \langle\dH(\xb),\dot{\xb}\rangle = \dH^\top(\xb)\Rb(\xb)\dH(\xb)\leq 0
    \end{equation}
    %
    Thus, if the conditions of this Theorem are satisfied, the system is stable in the sense of Lyapunov according to Theorem \ref{thm:hybrid_Lyap}. 
\end{proof}
%
Note that the condition (\ref{eq:stabhph_3}) is automatically satisfied if (\ref{eq:impulsive_ph}) is passive.
%
\begin{cor}\label{cor:impulsive_lyap}
	Consider a system in the form (\ref{eq:impulsive_ph}) and assume it to be passive. If
	%
	\begin{equation*}
	\exists \xb^*~:~\forall \xb\neq \xb^*~~\Ha(\xb)>0\text{ },\quad\Ha(\xb^*) = 0
	\end{equation*}
	% 
	then, for any $\epsilon>0$ there exists a Lyapunov stable set
	\[\A_\epsilon = \left\{\xb:\Ha(\xb)\leq\epsilon\right\}.\]
\end{cor}
%
Thus, every neighborhood of any strict minimum of $\Ha$ is stable as long as (\ref{eq:hph}) is passive.
%
\begin{rem}
    Regarding the impact pendulum of Example \ref{ex:ipend}, any neighborhood $\B_{o^+}$ of the origin,
    %
    \begin{equation}
        \B_{0^+}\triangleq \left\{(q,p)~:~0\leq q\leq\delta q\in\R^+\land0\leq p\leq\delta p\in\R^+\right\}
    \end{equation}
    %
    can be easily proven to be a Lyapunov stable set according to Theorem \ref{thm:impulsive_lyap}. Moreoever, thanks to Corollary \ref{cor:impulsive_lyap}, it can be deduced that for all nonnegative $\epsilon$, $\A_\epsilon$ is Lyapunov stable.
\end{rem}
%
\clearpage
%%%%%%%%%%%%%%%%%%%%%%%%%%%%%%%%%%%%%%%%%%%%%%%%%%%%%%%%%%%%%%%%%%%%%%%%%%%%%%%
%\section{Some Extensions of Passivity--Based Control}
%\clearpage
%%%%%%%%%%%%%%%%%%%%%%%%%%%%%%%%%%%%%%%%%%%%%%%%%%%%%%%%%%%%%%%%%%%%%%%%%%%%%%%
\section{Summary}
%
In this chapter, hybrid port--Hamiltonian systems have been defined. Both passivity and stability have also characterized.
%
\newline

%
As a final consideration, this chapter left opened the utility of this modeling framework for real applications, either in robotics or control theory.
%
\newline

%
Following, in the next chapters several applications are proposed, ranging from the control a ball--dribbling robot to the hybrid control of linear systems. Note that, in the latter case, explored in \ref{chap:multistable}, the ``hybrid'' nature is not given by the physics of the system but from the controller.
%