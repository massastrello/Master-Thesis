% This file was created by matlab2tikz.
%
%The latest updates can be retrieved from
%  http://www.mathworks.com/matlabcentral/fileexchange/22022-matlab2tikz-matlab2tikz
%where you can also make suggestions and rate matlab2tikz.
%
\begin{tikzpicture}

\begin{axis}[%
width=7cm,
height=3cm,
at={(0.858in,0.781in)},
scale only axis,
xmin=-0.8,
xmax=0.7,
xlabel style={font=\color{white!15!black}},
xlabel={$\xi(t)~[m]$},
ymin=-1.5,
ymax=1.5,
ylabel style={font=\color{white!15!black}},
ylabel={$\dot{\xi}(t)~[ms^{-1}]$},
axis background/.style={fill=white},
view={0}{90},
legend style={legend cell align=left, align=left, draw=white!15!black,font = \tiny}
]
\addplot3[contour filled={number = 25,labels={false}},mesh/rows=50,mesh/cols=50,mesh/check=false,forget plot
] table {H.dat};
\addplot [color=black, line width=2.0pt]
  table[row sep=crcr]{%
-0.8	0\\
-0.799994961191618	0.00499853361591898\\
-0.799979912367723	0.00994674685365824\\
-0.799954954570979	0.0148448771513189\\
-0.799920188365097	0.0196931622990959\\
-0.799602716042907	0.0431952443102433\\
-0.799054842548506	0.0654870877283772\\
-0.798288597012259	0.0865988632583716\\
-0.797315700639723	0.106560935957078\\
-0.79180232817447	0.173596516830792\\
-0.783973965721385	0.225640841163263\\
-0.774381288683226	0.26450204812555\\
-0.763501882011192	0.29197085945353\\
-0.747042020860754	0.314445976033828\\
-0.72976988716146	0.322423058231452\\
-0.712355838953056	0.319534738310389\\
-0.69531812887462	0.309027568468253\\
-0.676898302324055	0.291223211390291\\
-0.659689888119161	0.269623647126743\\
-0.643832734591055	0.246254610654186\\
-0.629420387260955	0.222725544977198\\
-0.61714861623052	0.201327892845499\\
-0.606081387447396	0.18112125032357\\
-0.596135540852008	0.162380668134624\\
-0.587228440080926	0.145270710313497\\
-0.577900541883002	0.127163976844156\\
-0.569737335894573	0.111269008881543\\
-0.562593329727953	0.0974262311004138\\
-0.556334145892474	0.085439774879882\\
-0.550553176429767	0.074552391514471\\
-0.54550217235343	0.0652463828839695\\
-0.541079788374854	0.0573124646460237\\
-0.537189412277445	0.0505290626374134\\
-0.533747712761737	0.0446944621100585\\
-0.530698125658422	0.0396842661615476\\
-0.527988925895292	0.0353797776615186\\
-0.525569970593653	0.0316571229862823\\
-0.52303674293993	0.0278748340815651\\
-0.520801579050654	0.0246558651069949\\
-0.518824975490308	0.0219151714733589\\
-0.517065812797719	0.0195526882398447\\
-0.515299213028131	0.0172373224513247\\
-0.513739062989125	0.0152555534744614\\
-0.512359606745628	0.013560631089518\\
-0.511132507072772	0.0120893290976102\\
-0.509841944435238	0.0105624552239169\\
-0.508712751315759	0.00925831671409115\\
-0.507725193949768	0.00814990412776904\\
-0.506855883378508	0.00718994555571561\\
-0.506014933876777	0.00626254189604836\\
-0.505281771732899	0.00546634770725144\\
-0.504643252864409	0.00478673008429337\\
-0.504084307800779	0.004197301320466\\
-0.503592377814803	0.00367704606621696\\
-0.503161214961675	0.00322497824440551\\
-0.502783544826694	0.00283339184190297\\
-0.502451790226188	0.00249118602720659\\
-0.502159488130832	0.00218923830571836\\
-0.501902551685047	0.00192511492921198\\
-0.501676794421557	0.00169449593215741\\
-0.501478101249754	0.00149211982928544\\
-0.501302911111012	0.00131355149543915\\
-0.501148668049079	0.0011567747013899\\
-0.501012909399428	0.00101928313453798\\
-0.500893294150882	0.000898347773725529\\
-0.500877951450443	0.000882825290388843\\
-0.500862873825328	0.000867574515221678\\
-0.500848056637506	0.000852590551720547\\
-0.50083349533071	0.000837868594377482\\
}
 [postaction={decorate, decoration={markings,
         mark=between positions 0.7 and 1 step 1 with {\arrow[black,line width=1.5pt]{latex};}
       }}]
;
\addlegendentry{$u = \beta(x)+v$}

\addplot [color=blue, dotted, line width=2.0pt, forget plot]
  table[row sep=crcr]{%
-0.50083349533071	0.000837868594377482\\
-0.50083349533071	0.000855510893233957\\
};
%\addlegendentry{data2}

\addplot [color=blue, line width=2.0pt]
  table[row sep=crcr]{%
-0.50083349533071	0.000855510893233957\\
-0.50083138815981	0.0033586205594724\\
-0.500826778512268	0.00586046219481607\\
-0.500819667662055	0.00836102392704575\\
-0.500810056895002	0.0108602938962747\\
-0.500797947508799	0.0133582602550021\\
-0.500783340812974	0.0158549111681654\\
-0.500766238128885	0.018350234813194\\
-0.500746640789707	0.0208442193800617\\
-0.500724550140417	0.0233368530713395\\
-0.500699967537785	0.0258281241022484\\
-0.500672894350357	0.0283180207007119\\
-0.500643331958447	0.0308065311074085\\
-0.500611281754122	0.0332936435758241\\
-0.500576745141187	0.0357793463723043\\
-0.500539723535175	0.0382636277761067\\
-0.500500218363335	0.0407464760794531\\
-0.500458231064615	0.0432278795875811\\
-0.500413763089652	0.0457078266187968\\
-0.500366815900759	0.0481863055045259\\
-0.50031739097191	0.0506633045893662\\
-0.500265489788728	0.0531388122311384\\
-0.500211113848472	0.0556128168009386\\
-0.500154264660022	0.0580853066831892\\
-0.500094943743869	0.0605562702756905\\
-0.500033152632097	0.063025695989672\\
-0.499968892868374	0.0654935722498436\\
-0.499902166007936	0.0679598874944468\\
-0.499832973617573	0.0704246301753054\\
-0.499761317275618	0.072887788757877\\
-0.499687198571931	0.0753493517213034\\
-0.499610619107885	0.0778093075584616\\
-0.499531580496354	0.0802676447760141\\
-0.499450084361699	0.0827243518944599\\
-0.499366132339754	0.0851794174481847\\
-0.499279726077809	0.0876328299855114\\
-0.499190867234601	0.0900845780687503\\
-0.499099557480297	0.0925346502742492\\
-0.499005798496481	0.0949830351924438\\
-0.49890959197614	0.0974297214279073\\
-0.498810939623647	0.0998746975994005\\
-0.498709843154752	0.102317952339922\\
-0.498606304296562	0.104759474296756\\
-0.498500324787532	0.107199252131526\\
-0.498391906377448	0.109637274520238\\
-0.49828105082741	0.112073530153339\\
-0.498167759909825	0.114508007735755\\
-0.498052035408383	0.116940695986949\\
-0.49793387911805	0.119371583640969\\
-0.497813292845051	0.121800659446492\\
-0.497690278406852	0.124227912166878\\
-0.497564837632153	0.126653330580217\\
-0.497436972360863	0.129076903479377\\
-0.497306684444096	0.131498619672054\\
-0.497173975744148	0.13391846798082\\
-0.497038848134484	0.136336437243172\\
-0.496901303499726	0.138752516311579\\
-0.496761343735636	0.14116669405353\\
-0.4966189707491	0.143578959351584\\
-0.496474186458115	0.145989301103419\\
-0.496326992791771	0.148397708221875\\
-0.49617739169024	0.150804169635006\\
-0.496025385104756	0.153208674286129\\
-0.495870974997602	0.155611211133865\\
-0.495714163342098	0.158011769152195\\
-0.495554952122578	0.1604103373305\\
-0.495393343334382	0.162806904673614\\
-0.495229338983836	0.165201460201868\\
-0.495062941088238	0.167593992951138\\
-0.494894151675843	0.169984491972892\\
-0.494722972785846	0.172372946334237\\
-0.494549406468367	0.174759345117966\\
-0.494373454784437	0.177143677422605\\
-0.494195119805977	0.179525932362458\\
-0.49401440361579	0.181906099067655\\
-0.493831308307538	0.184284166684199\\
-0.493645835985731	0.18666012437401\\
-0.493457988765707	0.189033961314976\\
-0.493267768773619	0.191405666700991\\
-0.49307517814642	0.19377522974201\\
-0.492880219031842	0.196142639664087\\
-0.492682893588384	0.198507885709428\\
-0.492483203985296	0.200870957136431\\
-0.492281152402559	0.203231843219735\\
-0.492076741030874	0.205590533250264\\
-0.491869972071641	0.207947016535272\\
-0.491660847736945	0.210301282398389\\
-0.491449370249538	0.212653320179668\\
-0.491235541842827	0.215003119235626\\
-0.491019364760851	0.217350668939293\\
-0.49080084125827	0.219695958680254\\
-0.490579973600343	0.222038977864696\\
-0.490356764062918	0.224379715915449\\
-0.490131214932411	0.226718162272036\\
-0.489903328505787	0.229054306390714\\
-0.489673107090551	0.231388137744516\\
-0.489440553004723	0.233719645823302\\
-0.489205668576827	0.236048820133797\\
-0.488968456145871	0.238375650199639\\
-0.48872891806133	0.240700125561417\\
-0.48848705668313	0.243022235776725\\
-0.488242874381633	0.245341970420195\\
-0.487996373537617	0.247659319083546\\
-0.487747556542257	0.24997427137563\\
-0.487496425797114	0.252286816922467\\
-0.487242983714113	0.254596945367299\\
-0.486987232715527	0.256904646370622\\
-0.48672917523396	0.259209909610238\\
-0.486468813712329	0.261512724781295\\
-0.486206150603849	0.263813081596328\\
-0.485941188372012	0.266110969785302\\
-0.485673929490572	0.268406379095659\\
-0.485404376443526	0.270699299292355\\
-0.485132531725098	0.272989720157905\\
-0.48485839783972	0.275277631492424\\
-0.484581977302018	0.277563023113674\\
-0.484303272636786	0.279845884857097\\
-0.484022286378978	0.282126206575867\\
-0.483739021073686	0.284403978140924\\
-0.483453479276118	0.28667918944102\\
-0.48316566355159	0.288951830382759\\
-0.482875576475498	0.291221890890639\\
-0.482583220633306	0.293489360907094\\
-0.482288598620527	0.295754230392535\\
-0.481991713042705	0.29801648932539\\
-0.481692566515396	0.300276127702147\\
-0.481391161664148	0.302533135537392\\
-0.481087501124491	0.304787502863854\\
-0.480781587541907	0.307039219732444\\
-0.480473423571822	0.309288276212293\\
-0.480163011879583	0.311534662390796\\
-0.479850355140439	0.313778368373652\\
-0.479535456039526	0.316019384284902\\
-0.479218317271846	0.318257700266972\\
-0.478898941542249	0.320493306480711\\
-0.478577331565415	0.322726193105433\\
-0.478253490065835	0.324956350338953\\
-0.477927419777795	0.327183768397634\\
-0.477599123445353	0.329408437516416\\
-0.477268603822325	0.331630347948867\\
-0.476935863672261	0.333849489967214\\
-0.476600905768431	0.336065853862386\\
-0.476263732893807	0.338279429944053\\
-0.475924347841039	0.340490208540664\\
-0.475582753412439	0.342698179999486\\
-0.475238952419964	0.344903334686644\\
-0.474892947685194	0.347105662987159\\
-0.474544742039317	0.349305155304987\\
-0.474194338323105	0.351501802063056\\
-0.473841739386897	0.353695593703307\\
-0.473486948090584	0.35588652068673\\
-0.473129967303583	0.358074573493404\\
-0.472770799904823	0.360259742622532\\
-0.472409448782724	0.362442018592484\\
-0.472045916835179	0.364621391940828\\
-0.471680206969531	0.366797853224375\\
-0.47131232210256	0.368971393019211\\
-0.470942265160457	0.371142001920738\\
-0.470570039078811	0.373309670543708\\
-0.470195646802585	0.375474389522265\\
-0.469819091286097	0.377636149509976\\
-0.469440375493005	0.379794941179874\\
-0.46905950239628	0.381950755224491\\
-0.468676474978194	0.384103582355895\\
-0.468291296230295	0.38625341330573\\
-0.467903969153392	0.388400238825249\\
-0.467514496757531	0.39054404968535\\
-0.467122882061978	0.392684836676617\\
-0.466729128095198	0.394822590609351\\
-0.466333237894836	0.396957302313608\\
-0.465935214507697	0.399088962639237\\
-0.465535060989727	0.401217562455912\\
-0.465132780405992	0.403343092653171\\
-0.464728375830658	0.405465544140451\\
-0.464321850346972	0.40758490784712\\
-0.463913207047241	0.409701174722518\\
-0.463502449032814	0.411814335735989\\
-0.46308957941406	0.413924381876915\\
-0.462674601310348	0.416031304154754\\
-0.462257517850029	0.418135093599074\\
-0.461838332170411	0.420235741259587\\
-0.461417047417748	0.422333238206182\\
-0.460993666747207	0.424427575528965\\
-0.460568193322861	0.426518744338288\\
-0.460140630317659	0.428606735764784\\
-0.459710980913409	0.430691540959406\\
-0.45927924830076	0.432773151093455\\
-0.458845435679179	0.434851557358616\\
-0.458409546256929	0.436926750966996\\
-0.457971583251053	0.43899872315115\\
-0.45753154988735	0.44106746516412\\
-0.457089449400357	0.443132968279469\\
-0.456645285033326	0.445195223791311\\
-0.456199060038205	0.447254223014346\\
-0.455750777675616	0.449309957283892\\
-0.455300441214836	0.45136241795592\\
-0.454848053933778	0.453411596407087\\
-0.454393619118963	0.455457484034765\\
-0.45393714006551	0.457500072257078\\
-0.453478620077105	0.459539352512932\\
-0.453018062465987	0.461575316262049\\
-0.452555470552923	0.463607954984997\\
-0.452090847667192	0.465637260183226\\
-0.451624197146558	0.467663223379095\\
-0.451155522337254	0.469685836115907\\
-0.450684826593958	0.471705089957943\\
-0.450212113279774	0.473720976490489\\
-0.449737385766211	0.475733487319869\\
-0.44926064743316	0.477742614073478\\
-0.448781901668874	0.479748348399812\\
-0.448301151869947	0.481750681968501\\
-0.447818401441293	0.483749606470338\\
-0.447333653796124	0.485745113617309\\
-0.446846912355932	0.487737195142626\\
-0.446358180550462	0.48972584280076\\
-0.445867461817695	0.491711048367466\\
-0.445374759603825	0.493692803639817\\
-0.444880077363241	0.495671100436232\\
-0.444383418558498	0.497645930596512\\
-0.443884786660306	0.499617285981861\\
-0.443384185147499	0.501585158474924\\
-0.442881617507019	0.503549539979812\\
-0.442377087233894	0.505510422422136\\
-0.441870597831214	0.507467797749032\\
-0.441362152810112	0.509421657929193\\
-0.440851755689741	0.511371994952898\\
-0.440339409997254	0.513318800832041\\
-0.439825119267781	0.515262067600162\\
-0.439308887044406	0.517201787312471\\
-0.438790716878149	0.519137952045884\\
-0.438270612327941	0.521070553899047\\
-0.437748576960605	0.522999584992363\\
-0.437224614350829	0.524925037468026\\
-0.436698728081153	0.526846903490045\\
-0.436170921741939	0.528765175244275\\
-0.435641198931352	0.530679844938444\\
-0.435109563255339	0.532590904802179\\
-0.434576018327607	0.534498347087039\\
-0.4340405677696	0.536402164066535\\
-0.433503215210478	0.538302348036167\\
-0.432963964287092	0.540198891313445\\
-0.432422818643969	0.542091786237917\\
-0.43187978193328	0.543981025171199\\
-0.431334857814829	0.545866600497\\
-0.430788049956021	0.54774850462115\\
-0.430239362031845	0.549626729971626\\
-0.429688797724852	0.551501268998579\\
-0.429136360725132	0.553372114174363\\
-0.428582054730289	0.555239257993557\\
-0.428025883445425	0.557102692972995\\
-0.42746785058311	0.558962411651792\\
-0.426907959863368	0.560818406591368\\
-0.426346215013646	0.562670670375477\\
-0.4257826197688	0.564519195610229\\
-0.425217177871066	0.566363974924119\\
-0.424649893070042	0.568205000968054\\
-0.424080769122663	0.570042266415374\\
-0.423509809793179	0.571875763961879\\
-0.422937018853133	0.573705486325856\\
-0.422362400081341	0.575531426248104\\
-0.421785957263863	0.577353576491958\\
-0.421207694193988	0.579171929843312\\
-0.420627614672205	0.580986479110648\\
-0.420045722506185	0.582797217125058\\
-0.419462021510755	0.584604136740269\\
-0.418876515507879	0.586407230832667\\
-0.418289208326631	0.588206492301322\\
-0.417700103803177	0.590001914068012\\
-0.417109205780748	0.591793489077247\\
-0.41651651810962	0.593581210296292\\
-0.415922044647089	0.595365070715194\\
-0.415325789257452	0.5971450633468\\
-0.414727755811979	0.598921181226787\\
-0.414127948188896	0.600693417413679\\
-0.413526370273356	0.602461764988876\\
-0.412923025957421	0.604226217056674\\
-0.412317919140037	0.605986766744288\\
-0.41171105372701	0.607743407201876\\
-0.411102433630985	0.609496131602561\\
-0.410492062771425	0.611244933142454\\
-0.409879945074581	0.612989805040677\\
-0.409266084473477	0.614730740539384\\
-0.408650484907881	0.616467732903787\\
-0.408033150324285	0.61820077542217\\
-0.407414084675882	0.619929861405922\\
-0.40679329192254	0.621654984189548\\
-0.406170776030784	0.623376137130701\\
-0.405546540973766	0.625093313610193\\
-0.404920590731249	0.626806507032028\\
-0.404292929289579	0.628515710823413\\
-0.403663560641663	0.630220918434786\\
-0.403032488786945	0.631922123339834\\
-0.402399717731387	0.633619319035516\\
-0.401765251487438	0.63531249904208\\
-0.401129094074018	0.63700165690309\\
-0.400491249516491	0.63868678618544\\
-0.399851721846643	0.64036788047938\\
-0.399210515102658	0.642044933398532\\
-0.398567633329092	0.643717938579915\\
-0.397923080576857	0.645386889683958\\
-0.397276860903189	0.647051780394529\\
-0.39662897837163	0.648712604418947\\
-0.395979437052004	0.650369355488005\\
-0.395328241020388	0.652022027355992\\
-0.394675394359099	0.653670613800708\\
-0.394020901156661	0.655315108623484\\
-0.393364765507785	0.656955505649206\\
-0.392706991513346	0.658591798726328\\
-0.392047583280358	0.660223981726894\\
-0.391386544921953	0.661852048546559\\
-0.390723880557353	0.6634759931046\\
-0.39005959431185	0.665095809343946\\
-0.389393690316782	0.666711491231184\\
-0.388726172709509	0.668323032756589\\
-0.388057045633385	0.669930427934133\\
-0.387386313237744	0.671533670801508\\
-0.386713979677866	0.673132755420143\\
-0.386040049114958	0.674727675875221\\
-0.385364525716134	0.676318426275699\\
-0.384687413654381	0.677905000754322\\
-0.384008717108547	0.679487393467643\\
-0.383328440263308	0.681065598596038\\
-0.382646587309149	0.682639610343728\\
-0.381963162442338	0.68420942293879\\
-0.381278169864905	0.685775030633178\\
-0.380591613784614	0.687336427702739\\
-0.379903498414941	0.688893608447227\\
-0.379213827975053	0.690446567190324\\
-0.378522606689779	0.691995298279653\\
-0.377829838789588	0.693539796086797\\
-0.377135528510568	0.695080055007312\\
-0.376439680094396	0.696616069460745\\
-0.375742297788321	0.698147833890649\\
-0.375043385845134	0.699675342764601\\
-0.374342948523147	0.701198590574215\\
-0.373640990086169	0.702717571835158\\
-0.37293751480348	0.704232281087167\\
-0.37223252694981	0.705742712894062\\
-0.371526030805311	0.707248861843762\\
-0.370818030655537	0.708750722548303\\
-0.370108530791418	0.710248289643847\\
-0.369397535509233	0.711741557790701\\
-0.368685049110591	0.713230521673331\\
-0.367971075902405	0.714715176000377\\
-0.367255620196865	0.716195515504663\\
-0.366538686311418	0.717671534943218\\
-0.36582027856874	0.719143229097286\\
-0.365100401296717	0.720610592772339\\
-0.364379058828412	0.722073620798095\\
-0.363656255502052	0.723532308028528\\
-0.362931995660995	0.724986649341882\\
-0.362206283653707	0.726436639640688\\
-0.361479123833743	0.727882273851771\\
-0.360750520559716	0.729323546926273\\
-0.360020478195278	0.730760453839652\\
-0.359289001109091	0.732192989591709\\
-0.358556093674807	0.733621149206594\\
-0.357821760271041	0.735044927732816\\
-0.357086005281347	0.736464320243264\\
-0.356348833094194	0.737879321835211\\
-0.355610248102942	0.73928992763033\\
-0.354870254705817	0.740696132774708\\
-0.354128857305887	0.742097932438855\\
-0.353386060311035	0.743495321817717\\
-0.35264186813394	0.744888296130688\\
-0.351896285192048	0.746276850621621\\
-0.351149315907548	0.747660980558842\\
-0.350400964707351	0.749040681235155\\
-0.349651236023059	0.750415947967863\\
-0.348900134290949	0.751786776098772\\
-0.348147663951941	0.753153160994202\\
-0.347393829451576	0.754515098045002\\
-0.346638635239994	0.755872582666559\\
-0.345882085771906	0.757225610298806\\
-0.345124185506572	0.758574176406238\\
-0.344364938907772	0.759918276477916\\
-0.34360435044379	0.761257906027483\\
-0.342842424587379	0.762593060593169\\
-0.342079165815744	0.763923735737808\\
-0.341314578610515	0.765249927048838\\
-0.340548667457721	0.766571630138322\\
-0.339781436847767	0.767888840642946\\
-0.33901289127541	0.769201554224039\\
-0.338243035239732	0.770509766567577\\
-0.337471873244117	0.77181347338419\\
-0.336699409796227	0.773112670409178\\
-0.335925649407976	0.774407353402515\\
-0.335150596595505	0.775697518148856\\
-0.334374255879158	0.776983160457554\\
-0.333596631783458	0.778264276162659\\
-0.332817728837081	0.779540861122932\\
-0.332037551572833	0.780812911221853\\
-0.331256104527622	0.782080422367629\\
-0.330473392242438	0.783343390493199\\
-0.329689419262324	0.784601811556246\\
-0.328904190136352	0.785855681539203\\
-0.328117709417603	0.787104996449261\\
-0.327329981663135	0.788349752318377\\
-0.326541011433961	0.78958994520328\\
-0.325750803295029	0.79082557118548\\
-0.32495936181519	0.792056626371276\\
-0.324166691567175	0.793283106891758\\
-0.323372797127576	0.794505008902821\\
-0.322577683076813	0.795722328585167\\
-0.321781353999114	0.796935062144313\\
-0.320983814482489	0.798143205810598\\
-0.320185069118705	0.799346755839188\\
-0.319385122503263	0.800545708510086\\
-0.31858397923537	0.801740060128132\\
-0.317781643917915	0.802929807023016\\
-0.316978121157447	0.804114945549278\\
-0.316173415564147	0.805295472086318\\
-0.315367531751803	0.806471383038399\\
-0.314560474337789	0.807642674834654\\
-0.313752247943034	0.808809343929091\\
-0.312942857192005	0.809971386800599\\
-0.312132306712673	0.81112879995295\\
-0.311320601136496	0.812281579914809\\
-0.310507745098389	0.813429723239736\\
-0.309693743236703	0.81457322650619\\
-0.308878600193196	0.815712086317537\\
-0.308062320613011	0.816846299302048\\
-0.30724490914465	0.817975862112913\\
-0.30642637043995	0.819100771428237\\
-0.305606709154058	0.820221023951048\\
-0.304785929945403	0.821336616409299\\
-0.303964037475677	0.822447545555876\\
-0.303141036409803	0.823553808168596\\
-0.302316931415917	0.824655401050214\\
-0.301491727165337	0.825752321028427\\
-0.300665428332544	0.826844564955877\\
-0.29983803959515	0.827932129710152\\
-0.299009565633881	0.82901501219379\\
-0.298180011132544	0.830093209334287\\
-0.297349380778009	0.831166718084091\\
-0.296517679260181	0.832235535420613\\
-0.295684911271972	0.833299658346223\\
-0.294851081509283	0.834359083888259\\
-0.294016194670973	0.835413809099024\\
-0.293180255458837	0.836463831055789\\
-0.292343268577579	0.8375091468608\\
-0.29150523873479	0.838549753641272\\
-0.290666170640921	0.8395856485494\\
-0.289826069009256	0.840616828762352\\
-0.288984938555894	0.841643291482276\\
-0.288142783999714	0.842665033936301\\
-0.287299610062359	0.843682053376538\\
-0.286455421468207	0.844694347080077\\
-0.285610222944344	0.845701912348996\\
-0.284764019220545	0.846704746510356\\
-0.283916815029243	0.847702846916204\\
-0.283068615105506	0.848696210943574\\
-0.282219424187015	0.849684835994487\\
-0.281369247014034	0.850668719495951\\
-0.280518088329389	0.851647858899962\\
-0.279665952878441	0.852622251683504\\
-0.278812845409061	0.853591895348552\\
-0.277958770671608	0.854556787422065\\
-0.277103733418898	0.855516925455994\\
-0.276247738406184	0.856472307027275\\
-0.27539079039113	0.857422929737834\\
-0.274532894133786	0.858368791214584\\
-0.273674054396562	0.859309889109422\\
-0.272814275944203	0.860246221099232\\
-0.271953563543765	0.861177784885885\\
-0.27109192196459	0.862104578196232\\
-0.270229355978281	0.86302659878211\\
-0.269365870358675	0.863943844420335\\
-0.268501469881822	0.864856312912705\\
-0.267636159325956	0.865764002085996\\
-0.266769943471473	0.866666909791959\\
-0.265902827100903	0.867565033907323\\
-0.265034814998888	0.86845837233379\\
-0.264165911952157	0.869346922998031\\
-0.263296122749496	0.870230683851689\\
-0.262425452181731	0.871109652871371\\
-0.261553905041698	0.871983828058653\\
-0.260681486124216	0.872853207440068\\
-0.259808200226069	0.873717789067113\\
-0.258934052145975	0.874577571016238\\
-0.258059046684562	0.87543255138885\\
-0.257183188644348	0.876282728311305\\
-0.256306482829707	0.877128099934908\\
-0.255428934046852	0.877968664435908\\
-0.254550547103809	0.878804420015494\\
-0.253671326810387	0.879635364899796\\
-0.252791277978158	0.880461497339875\\
-0.25191040542043	0.881282815611723\\
-0.251028713952223	0.882099318016261\\
-0.250146208390245	0.882911002879329\\
-0.249262893552862	0.88371786855169\\
-0.24837877426008	0.884519913409017\\
-0.247493855333516	0.885317135851895\\
-0.246608141596373	0.886109534305814\\
-0.245721637873417	0.886897107221166\\
-0.244834348990952	0.887679853073238\\
-0.243946279776792	0.88845777036221\\
-0.24305743506024	0.889230857613146\\
-0.242167819672062	0.889999113375995\\
-0.24127743844446	0.890762536225578\\
-0.240386296211049	0.89152112476159\\
-0.239494397806833	0.89227487760859\\
-0.238601748068179	0.893023793415999\\
-0.23770835183279	0.893767870858089\\
-0.236814213939685	0.894507108633983\\
-0.23591933922917	0.895241505467646\\
-0.235023732542814	0.89597106010788\\
-0.234127398723427	0.896695771328315\\
-0.233230342615032	0.897415637927408\\
-0.232332569062839	0.898130658728431\\
-0.231434082913225	0.89884083257947\\
-0.230534889013706	0.899546158353414\\
-0.229634992212912	0.900246634947949\\
-0.228734397360563	0.900942261285552\\
-0.227833109307444	0.901633036313486\\
-0.226931132905382	0.902318959003788\\
-0.226028473007217	0.903000028353266\\
-0.225125134466782	0.90367624338349\\
-0.224221122138875	0.904347603140783\\
-0.223316440879235	0.905014106696219\\
-0.222411095544518	0.905675753145605\\
-0.221505090992272	0.906332541609487\\
-0.220598432080913	0.906984471233127\\
-0.219691123669696	0.907631541186508\\
-0.218783170618698	0.908273750664317\\
-0.217874577788787	0.908911098885942\\
-0.216965350041597	0.909543585095459\\
-0.21605549223951	0.910171208561628\\
-0.215145009245624	0.910793968577881\\
-0.214233905923732	0.911411864462315\\
-0.213322187138296	0.912024895557682\\
-0.212409857754424	0.91263306123138\\
-0.211496922637844	0.913236360875445\\
-0.210583386654878	0.913834793906541\\
-0.209669254672422	0.91442835976595\\
-0.208754531557915	0.915017057919564\\
-0.207839222179321	0.915600887857873\\
-0.2069233314051	0.916179849095957\\
-0.206006864104183	0.916753941173477\\
-0.205089825145952	0.917323163654663\\
-0.20417221940021	0.917887516128304\\
-0.20325405173716	0.91844699820774\\
-0.20233532702738	0.919001609530849\\
-0.201416050141797	0.919551349760039\\
-0.200496225951662	0.920096218582234\\
-0.199575859328531	0.920636215708868\\
-0.198654955144232	0.921171340875869\\
-0.197733518270846	0.921701593843652\\
-0.196811553580683	0.922226974397108\\
-0.195889065946254	0.922747482345589\\
-0.194966060240249	0.923263117522901\\
-0.194042541335512	0.92377387978729\\
-0.193118514105018	0.924279769021431\\
-0.192193983421844	0.924780785132418\\
-0.191268954159151	0.92527692805175\\
-0.190343431190153	0.92576819773532\\
-0.189417419388099	0.926254594163404\\
-0.188490923626243	0.926736117340647\\
-0.187563948777825	0.927212767296052\\
-0.186636499716041	0.927684544082968\\
-0.185708581314023	0.928151447779078\\
-0.184780198444813	0.928613478486382\\
-0.183851355981338	0.929070636331193\\
-0.182922058796388	0.929522921464114\\
-0.181992311762589	0.929970334060034\\
-0.181062119752381	0.930412874318109\\
-0.180131487637991	0.930850542461752\\
-0.179200420291413	0.931283338738618\\
-0.178268922584379	0.931711263420593\\
-0.177336999388339	0.932134316803776\\
-0.176404655574431	0.932552499208473\\
-0.175471896013466	0.932965810979173\\
-0.174538725575895	0.933374252484545\\
-0.173605149131788	0.933777824117415\\
-0.172671171550812	0.93417652629476\\
-0.171736797702204	0.934570359457685\\
-0.170802032454747	0.934959324071418\\
-0.169866880676748	0.935343420625287\\
-0.168931347236012	0.935722649632713\\
-0.167995436999819	0.936097011631191\\
-0.167059154834899	0.936466507182274\\
-0.166122505607409	0.936831136871563\\
-0.165185494182907	0.93719090130869\\
-0.164248125426331	0.9375458011273\\
-0.163310404201973	0.93789583698504\\
-0.162372335373455	0.938241009563542\\
-0.161433923803704	0.938581319568405\\
-0.160495174354933	0.938916767729186\\
-0.15955609188861	0.939247354799378\\
-0.158616681265439	0.939573081556397\\
-0.157676947345334	0.939893948801566\\
-0.156736894987398	0.9402099573601\\
-0.155796529049894	0.940521108081087\\
-0.154855854390224	0.940827401837475\\
-0.153914875864908	0.941128839526054\\
-0.152973598329555	0.94142542206744\\
-0.152032026638841	0.941717150406057\\
-0.151090165646488	0.942004025510126\\
-0.150148020205235	0.942286048371638\\
-0.14920559516682	0.942563220006349\\
-0.148262895381952	0.942835541453754\\
-0.147319925700288	0.943103013777074\\
-0.146376690970411	0.943365638063239\\
-0.145433196039804	0.943623415422869\\
-0.144489445754829	0.943876346990255\\
-0.143545444960702	0.94412443392335\\
-0.142601198501467	0.944367677403739\\
-0.141656711219977	0.944606078636632\\
-0.140711987957867	0.944839638850839\\
-0.139767033555531	0.945068359298754\\
-0.1388218528521	0.945292241256342\\
-0.137876450685416	0.945511286023112\\
-0.13693083189201	0.945725494922107\\
-0.135985001307079	0.945934869299878\\
-0.135038963764461	0.946139410526473\\
-0.134092724096612	0.946339119995414\\
-0.133146287134584	0.946533999123677\\
-0.132199657707998	0.946724049351679\\
-0.131252840645026	0.946909272143251\\
-0.130305840772362	0.947089668985628\\
-0.129358662915201	0.947265241389422\\
-0.128411311897218	0.947435990888607\\
-0.12746379254054	0.947601919040499\\
-0.126516109665726	0.947763027425736\\
-0.125568268091743	0.947919317648259\\
-0.124620272635942	0.948070791335292\\
-0.123672128114034	0.948217450137321\\
-0.122723839340071	0.948359295728077\\
-0.121775411126417	0.948496329804513\\
-0.120826848283727	0.948628554086787\\
-0.119878155620927	0.948755970318239\\
-0.118929337945185	0.948878580265371\\
-0.117980400061894	0.948996385717829\\
-0.117031346774644	0.949109388488379\\
-0.1160821828852	0.949217590412891\\
-0.115132913193481	0.949320993350313\\
-0.114183542497535	0.949419599182653\\
-0.113234075593516	0.949513409814959\\
-0.112284517275662	0.949602427175296\\
-0.111334872336272	0.949686653214725\\
-0.110385145565682	0.949766089907283\\
-0.109435341752242	0.949840739249962\\
-0.108485465682292	0.949910603262683\\
-0.107535522140145	0.949975683988283\\
-0.106585515908056	0.950035983492484\\
-0.105635451766205	0.950091503863877\\
-0.104685334492669	0.9501422472139\\
-0.103735168863406	0.950188215676812\\
-0.102784959652226	0.950229411409676\\
-0.101834711630771	0.950265836592332\\
-0.100884429568491	0.950297493427381\\
-0.0999341182326237	0.950324384140155\\
-0.098983782388169	0.950346510978699\\
-0.0980334267978677	0.950363876213749\\
-0.0970830562221785	0.950376482138707\\
-0.0961326754192551	0.950384331069619\\
-0.0951822891449244	0.950387425345154\\
-0.0942319021526632	0.950385767326576\\
-0.0932815191935751	0.950379359397727\\
-0.0923311450163692	0.950368203965001\\
-0.0913807843673368	0.950352303457319\\
-0.0904304419903294	0.950331660326108\\
-0.0894801226267348	0.950306277045277\\
-0.0885298310154567	0.950276156111193\\
-0.0875795718928907	0.950241300042658\\
-0.0866293499929027	0.950201711380883\\
-0.0856791700468059	0.950157392689467\\
-0.0847290367833387	0.950108346554373\\
-0.0837789549286425	0.950054575583901\\
-0.0828289292062394	0.949996082408664\\
-0.0818789643370093	0.949932869681568\\
-0.0809290650391687	0.949864940077785\\
-0.0799792360282472	0.949792296294725\\
-0.0790294820170659	0.949714941052016\\
-0.078079807715716	0.949632877091479\\
-0.0771302178315346	0.949546107177103\\
-0.0761807170690847	0.949454634095015\\
-0.0752313101301312	0.949358460653464\\
-0.074282001713621	0.949257589682789\\
-0.0733327965156584	0.949152024035395\\
-0.0723836992294845	0.949041766585731\\
-0.0714347145454556	0.948926820230259\\
-0.0704858471510201	0.948807187887436\\
-0.0695371017306966	0.94868287249768\\
-0.068588482966053	0.948553877023352\\
-0.0676399955356836	0.948420204448725\\
-0.0666916441151873	0.948281857779959\\
-0.0657434333771459	0.948138840045079\\
-0.0647953679911028	0.947991154293942\\
-0.0638474526235399	0.947838803598218\\
-0.0628996919378569	0.947681791051358\\
-0.0619520905943494	0.947520119768571\\
-0.0610046532501862	0.947353792886799\\
-0.0600573845593895	0.947182813564683\\
-0.0591102891728104	0.947007184982545\\
-0.0581633717381106	0.946826910342358\\
-0.0572166368997377	0.946641992867717\\
-0.0562700892989052	0.946452435803815\\
-0.0553237335735711	0.946258242417415\\
-0.0543775743584153	0.946059415996823\\
-0.0534316162848192	0.94585595985186\\
-0.0524858639808429	0.945647877313835\\
-0.0515403220712049	0.945435171735518\\
-0.0505949951772607	0.945217846491115\\
-0.04964988791698	0.944995904976235\\
-0.0487050049049264	0.944769350607865\\
-0.0477603507522364	0.944538186824345\\
-0.0468159300665964	0.944302417085334\\
-0.045871747452224	0.94406204487179\\
-0.044927807509844	0.943817073685934\\
-0.0439841148366686	0.943567507051227\\
-0.0430406740263765	0.943313348512339\\
-0.0420974896690902	0.943054601635123\\
-0.0411545663513569	0.942791270006586\\
-0.0402119086561252	0.942523357234859\\
-0.0392695211627254	0.942250866949168\\
-0.0383274084468481	0.94197380279981\\
-0.0373855750805235	0.941692168458118\\
-0.0364440256320989	0.941405967616437\\
-0.0355027646662197	0.941115203988092\\
-0.034561796743807	0.940819881307359\\
-0.0336211264220374	0.94052000332944\\
-0.0326807582543221	0.940215573830428\\
-0.031740696790285	0.93990659660728\\
-0.0308009465757431	0.93959307547779\\
-0.0298615121526855	0.939275014280557\\
-0.0289223980592517	0.938952416874954\\
-0.0279836088297118	0.938625287141103\\
-0.0270451489944456	0.938293628979839\\
-0.026107023079921	0.937957446312687\\
-0.025169235608675	0.937616743081826\\
-0.0242317910992912	0.937271523250062\\
-0.023294694066381	0.936921790800799\\
-0.022357949020561	0.936567549738006\\
-0.0214215604684348	0.936208804086188\\
-0.0204855329125704	0.935845557890355\\
-0.0195498708514806	0.935477815215995\\
-0.0186145787796028	0.935105580149035\\
-0.0176796611872781	0.934728856795823\\
-0.0167451225607312	0.934347649283082\\
-0.0158109673820496	0.933961961757895\\
-0.014877200129164	0.933571798387662\\
-0.0139438252758269	0.933177163360075\\
-0.0130108472915937	0.932778060883084\\
-0.0120782706418014	0.932374495184869\\
-0.0111460997875486	0.931966470513806\\
-0.0102143391856755	0.931553991138436\\
-0.00928299328874372	0.931137061347435\\
-0.00835206654501644	0.930715685449583\\
-0.00742156339843743	0.930289867773728\\
-0.00649148828861231	0.92985961266876\\
-0.00556184565078693	0.929424924503576\\
-0.00463263991582918	0.928985807667048\\
-0.00370387551020739	0.928542266567993\\
-0.00277555685597122	0.92809430563514\\
-0.00184768837073223	0.927641929317097\\
-0.000920274467642927	0.927185142082321\\
6.68044462262888e-06	0.926723948419081\\
0.000933171961888126	0.926258352835434\\
0.00185919568449575	0.925788359859184\\
0.00278474721732472	0.925313974037854\\
0.00370982216981178	0.924835199938654\\
0.00463441615597144	0.924352042148443\\
0.0055585247944138	0.923864505273706\\
0.00648214370836574	0.92337259394051\\
0.00740526852569016	0.922876312794479\\
0.00832789487890546	0.922375666500757\\
0.00925001840520382	0.921870659743978\\
0.0101716347464734	0.92136129722823\\
0.0110927395493152	0.920847583677022\\
0.0120133284650631	0.920329523833254\\
0.0129333971498045	0.919807122459179\\
0.0138529412643983	0.919280384336372\\
0.0147719564744945	0.918749314265698\\
0.0156904384505535	0.918213917067273\\
0.016608382867866	0.917674197580438\\
0.0175257854065708	0.917130160663717\\
0.0184426417516757	0.91658181119479\\
0.0193589475930751	0.916029154070455\\
0.0202746986255699	0.915472194206594\\
0.0211898905488866	0.914910936538142\\
0.0221045190676959	0.91434538601905\\
0.0230185798916319	0.913775547622251\\
0.0239320687353115	0.913201426339626\\
0.0248449813183527	0.912623027181972\\
0.0257573133653935	0.912040355178962\\
0.0266690606061114	0.911453415379116\\
0.0275802187752418	0.910862212849763\\
0.0284907836125964	0.910266752677008\\
0.0294007508630828	0.909667039965695\\
0.0303101162767225	0.909063079839376\\
0.0312188756086704	0.90845487744027\\
0.0321270246192321	0.907842437929236\\
0.0330345590738844	0.907225766485731\\
0.0339414747432919	0.906604868307777\\
0.034847767403327	0.905979748611926\\
0.0357534328350879	0.905350412633227\\
0.0366584668249169	0.904716865625184\\
0.0375628651644191	0.904079112859729\\
0.0384666236504805	0.903437159627181\\
0.0393697380852871	0.902791011236209\\
0.0402722042763421	0.902140673013803\\
0.0411740180364855	0.901486150305232\\
0.0420751751839116	0.900827448474009\\
0.0429756715421865	0.900164572901858\\
0.0438755029402681	0.899497528988676\\
0.0447746652125227	0.898826322152497\\
0.0456731541987439	0.898150957829456\\
0.0465709657441705	0.897471441473754\\
0.0474680956995043	0.896787778557619\\
0.0483645399209289	0.89609997457127\\
0.0492602942701261	0.895408035022886\\
0.0501553546142957	0.89471196543856\\
0.0510497168261723	0.894011771362273\\
0.0519433767840434	0.893307458355849\\
0.0528363303717672	0.89259903199892\\
0.0537285734787899	0.891886497888893\\
0.0546201020001652	0.891169861640909\\
0.0555109118365698	0.89044912888781\\
0.0564009988943225	0.889724305280096\\
0.0572903590854011	0.888995396485896\\
0.0581789883274605	0.888262408190922\\
0.0590668825438503	0.887525346098439\\
0.0599540376636316	0.886784215929226\\
0.0608404496215958	0.886039023421534\\
0.0617261143582802	0.885289774331055\\
0.0626110278199877	0.88453647443088\\
0.063495185958802	0.883779129511466\\
0.0643785847326066	0.883017745380591\\
0.065261220105101	0.882252327863326\\
0.0661430880458184	0.881482882801988\\
0.0670241845301432	0.880709416056108\\
0.0679045055393278	0.879931933502391\\
0.0687840470605097	0.879150441034678\\
0.0696628050867292	0.87836494456391\\
0.0705407756169459	0.877575450018086\\
0.0714179546560553	0.87678196334223\\
0.0722943382149071	0.875984490498346\\
0.073169922310322	0.875183037465387\\
0.0740447029651068	0.874377610239211\\
0.0749186762080734	0.873568214832547\\
0.0757918380740547	0.872754857274952\\
0.0766641846039215	0.871937543612776\\
0.0775357118445994	0.871116279909123\\
0.0784064158490856	0.870291072243809\\
0.0792762926764652	0.869461926713327\\
0.0801453383919281	0.868628849430809\\
0.081013549066786	0.867791846525982\\
0.0818809207784881	0.866950924145134\\
0.0827474496106383	0.866106088451071\\
0.083613131653012	0.865257345623082\\
0.0844779630015715	0.864404701856898\\
0.0853419397584835	0.863548163364651\\
0.0862050580321349	0.862687736374837\\
0.0870673139371488	0.861823427132278\\
0.0879287035944021	0.860955241898078\\
0.0887892231310399	0.86008318694959\\
0.0896488686804942	0.859207268580368\\
0.0905076363824969	0.858327493100136\\
0.091365522383099	0.857443866834743\\
0.0922225228346851	0.856556396126126\\
0.0930786338959894	0.855665087332268\\
0.0939338517321123	0.854769946827162\\
0.0947881725145364	0.853870981000764\\
0.0956415924211421	0.852968196258961\\
0.0964941076362229	0.852061599023528\\
0.0973457143505036	0.851151195732087\\
0.0981964087611527	0.850236992838065\\
0.0990461870718016	0.84931899681066\\
0.0998950454925577	0.848397214134795\\
0.100742980240022	0.84747165131108\\
0.101589987537303	0.846542314855773\\
0.102436063614033	0.845609211300736\\
0.103281204706386	0.844672347193398\\
0.10412540705709	0.843731729096713\\
0.104968666915443	0.84278736358912\\
0.105810980537329	0.841839257264501\\
0.106652344185236	0.840887416732141\\
0.107492754128266	0.839931848616688\\
0.108332206642155	0.838972559558114\\
0.109170698009286	0.838009556211668\\
0.110008224518705	0.837042845247842\\
0.110844782466137	0.836072433352326\\
0.111680368153999	0.835098327225967\\
0.112514977891418	0.83412053358473\\
0.113348607994241	0.833139059159657\\
0.114181254785057	0.832153910696822\\
0.115012914593209	0.831165094957294\\
0.115843583754805	0.830172618717094\\
0.116673258612739	0.829176488767153\\
0.117501935516704	0.828176711913274\\
0.118329610823205	0.827173294976083\\
0.119156280895574	0.826166244790997\\
0.119981942103987	0.825155568208176\\
0.120806590825479	0.824141272092483\\
0.121630223443954	0.823123363323444\\
0.122452836350206	0.822101848795201\\
0.123274425941928	0.82107673541648\\
0.12409498862373	0.820048030110537\\
0.124914520807152	0.819015739815128\\
0.125733018910681	0.817979871482457\\
0.12655047935976	0.816940432079142\\
0.127366898586809	0.815897428586166\\
0.128182273031234	0.814850867998841\\
0.128996599139445	0.813800757326763\\
0.129809873364867	0.812747103593768\\
0.130622092167958	0.811689913837895\\
0.131433252016219	0.810629195111338\\
0.132243349384212	0.809564954480407\\
0.133052380753572	0.808497199025484\\
0.13386034261302	0.807425935840984\\
0.13466723145838	0.806351172035307\\
0.135473043792591	0.805272914730801\\
0.136277776125721	0.804191171063713\\
0.137081424974981	0.803105948184155\\
0.137883986864739	0.802017253256054\\
0.138685458326536	0.800925093457111\\
0.139485835899093	0.79982947597876\\
0.140285116128335	0.798730408026126\\
0.141083295567394	0.797627896817978\\
0.14188037077663	0.796521949586688\\
0.142676338323641	0.795412573578192\\
0.14347119478328	0.794299776051939\\
0.144264936737663	0.793183564280857\\
0.145057560776189	0.792063945551302\\
0.145849063495546	0.790940927163019\\
0.146639441499733	0.7898145164291\\
0.147428691400064	0.788684720675936\\
0.148216809815191	0.787551547243179\\
0.149003793371109	0.786415003483695\\
0.149789638701172	0.785275096763521\\
0.150574342446108	0.784131834461826\\
0.151357901254032	0.782985223970859\\
0.152140311780455	0.781835272695915\\
0.152921570688302	0.780681988055284\\
0.153701674647921	0.779525377480213\\
0.154480620337099	0.778365448414858\\
0.155258404441073	0.777202208316242\\
0.156035023652543	0.776035664654213\\
0.156810474671687	0.774865824911397\\
0.157584754206171	0.773692696583157\\
0.158357858971161	0.772516287177548\\
0.15912978568934	0.771336604215272\\
0.159900531090917	0.770153655229638\\
0.16067009191364	0.768967447766512\\
0.16143846490281	0.767777989384278\\
0.162205646811292	0.766585287653792\\
0.162971634399529	0.765389350158339\\
0.163736424435553	0.764190184493588\\
0.164500013694996	0.762987798267545\\
0.165262398961107	0.761782199100516\\
0.166023577024761	0.760573394625056\\
0.166783544684469	0.759361392485928\\
0.167542298746397	0.758146200340057\\
0.16829983602437	0.756927825856487\\
0.169056153339891	0.755706276716337\\
0.169811247522148	0.754481560612756\\
0.17056511540803	0.753253685250874\\
0.171317753842136	0.752022658347768\\
0.172069159676788	0.750788487632406\\
0.172819329772043	0.749551180845612\\
0.173568260995706	0.748310745740013\\
0.174315950223339	0.747067190080002\\
0.175062394338273	0.745820521641685\\
0.175807590231625	0.744570748212845\\
0.176551534802302	0.743317877592893\\
0.177294224957017	0.74206191759282\\
0.178035657610302	0.740802876035159\\
0.178775829684513	0.739540760753935\\
0.17951473810985	0.738275579594621\\
0.180252379824361	0.737007340414097\\
0.18098875177396	0.735736051080598\\
0.181723850912432	0.734461719473674\\
0.182457674201447	0.733184353484146\\
0.183190218610573	0.731903961014056\\
0.183921481117286	0.730620549976626\\
0.184651458706978	0.729334128296209\\
0.185380148372975	0.728044703908251\\
0.186107547116541	0.726752284759237\\
0.186833651946892	0.725456878806649\\
0.187558459881209	0.724158494018927\\
0.188281967944645	0.722857138375413\\
0.189004173170338	0.721552819866313\\
0.189725072599423	0.720245546492649\\
0.190444663281039	0.718935326266214\\
0.191162942272345	0.717622167209529\\
0.191879906638525	0.71630607735579\\
0.192595553452805	0.714987064748833\\
0.193309879796457	0.713665137443081\\
0.194022882758813	0.7123403035035\\
0.194734559437277	0.711012571005556\\
0.195444906937333	0.709681948035165\\
0.196153922372556	0.708348442688652\\
0.196861602864621	0.707012063072702\\
0.197567945543317	0.705672817304315\\
0.198272947546554	0.704330713510762\\
0.198976606020376	0.702985759829536\\
0.199678918118967	0.701637964408311\\
0.200379881004665	0.700287335404891\\
0.201079491847972	0.698933880987166\\
0.20177774782756	0.69757760933307\\
0.202474646130288	0.696218528630528\\
0.203170183951204	0.694856647077415\\
0.203864358493562	0.69349197288151\\
0.204557166968826	0.692124514260448\\
0.205248606596683	0.690754279441674\\
0.205938674605055	0.689381276662399\\
0.206627368230103	0.688005514169552\\
0.207314684716242	0.686627000219736\\
0.208000621316146	0.68524574307918\\
0.208685175290762	0.683861751023691\\
0.209368343909317	0.682475032338614\\
0.210050124449328	0.68108559531878\\
0.210730514196611	0.679693448268462\\
0.211409510445292	0.67829859950133\\
0.212087110497816	0.676901057340401\\
0.212763311664952	0.675500830117997\\
0.213438111265811	0.674097926175698\\
0.214111506627847	0.67269235386429\\
0.21478349508687	0.671284121543729\\
0.215454073987054	0.669873237583083\\
0.216123240680949	0.668459710360496\\
0.216790992529484	0.667043548263135\\
0.217457326901982	0.665624759687145\\
0.218122241176167	0.664203353037605\\
0.218785732738171	0.662779336728477\\
0.219447798982545	0.661352719182564\\
0.220108437312267	0.659923508831461\\
0.220767645138752	0.65849171411551\\
0.22142541988186	0.65705734348375\\
0.222081758969901	0.655620405393876\\
0.22273665983965	0.654180908312187\\
0.223390119936353	0.652738860713542\\
0.224042136713732	0.651294271081315\\
0.224692707634001	0.649847147907343\\
0.225341830167867	0.648397499691886\\
0.225989501794541	0.646945334943574\\
0.226635720001748	0.645490662179366\\
0.227280482285736	0.644033489924498\\
0.227923786151278	0.642573826712441\\
0.228565629111689	0.641111681084851\\
0.229206008688826	0.639647061591524\\
0.229844922413102	0.638179976790345\\
0.230482367823493	0.636710435247249\\
0.231118342467541	0.635238445536168\\
0.231752843901369	0.633764016238986\\
0.232385869689687	0.632287155945491\\
0.233017417405794	0.630807873253331\\
0.233647484631595	0.629326176767963\\
0.234276068957603	0.627842075102611\\
0.234903167982947	0.626355576878213\\
0.235528779315383	0.62486669072338\\
0.236152900571296	0.623375425274346\\
0.236775529375714	0.621881789174921\\
0.237396663362313	0.620385791076444\\
0.238016300173421	0.618887439637737\\
0.23863443746003	0.617386743525059\\
0.239251072881804	0.615883711412054\\
0.239866204107081	0.61437835197971\\
0.240479828812886	0.612870673916308\\
0.241091944684935	0.611360685917378\\
0.241702549417643	0.609848396685647\\
0.242311640714132	0.608333814930997\\
0.242919216286237	0.606816949370417\\
0.243525273854512	0.605297808727952\\
0.244129811148242	0.60377640173466\\
0.244732825905444	0.602252737128565\\
0.245334315872876	0.600726823654605\\
0.245934278806048	0.599198670064592\\
0.24653271246922	0.597668285117157\\
0.247129614635418	0.59613567757771\\
0.247724983086435	0.594600856218388\\
0.248318815612841	0.593063829818009\\
0.248911110013986	0.591524607162026\\
0.24950186409801	0.58998319704248\\
0.250091075681849	0.588439608257949\\
0.250678742591239	0.586893849613504\\
0.251264862660727	0.585345929920665\\
0.251849433733671	0.583795857997343\\
0.252432453662252	0.582243642667806\\
0.253013920307481	0.580689292762622\\
0.253593831539198	0.579132817118615\\
0.254172185236086	0.577574224578819\\
0.254748979285673	0.576013523992429\\
0.25532421158434	0.574450724214754\\
0.255897880037326	0.572885834107169\\
0.256469982558734	0.57131886253707\\
0.257040517071537	0.569749818377824\\
0.257609481507586	0.568178710508724\\
0.258176873807613	0.566605547814938\\
0.258742691921238	0.565030339187466\\
0.259306933806976	0.563453093523091\\
0.259869597432239	0.56187381972433\\
0.260430680773349	0.56029252669939\\
0.260990181815534	0.558709223362115\\
0.261548098552942	0.557123918631946\\
0.262104428988642	0.555536621433868\\
0.262659171134631	0.553947340698366\\
0.263212323011838	0.552356085361372\\
0.263763882650133	0.550762864364228\\
0.264313848088329	0.549167686653626\\
0.264862217374186	0.547570561181572\\
0.265408988564424	0.54597149690533\\
0.265954159724717	0.54437050278738\\
0.266497728929707	0.542767587795367\\
0.267039694263007	0.541162760902058\\
0.267580053817203	0.539556031085288\\
0.268118805693864	0.53794740732792\\
0.268655948003542	0.536336898617793\\
0.269191478865779	0.534724513947674\\
0.269725396409116	0.533110262315213\\
0.27025769877109	0.531494152722896\\
0.270788384098244	0.529876194177996\\
0.271317450546132	0.528256395692525\\
0.271844896279321	0.526634766283188\\
0.272370719471398	0.525011314971335\\
0.272894918304972	0.523386050782914\\
0.273417490971681	0.521758982748422\\
0.273938435672198	0.520130119902861\\
0.27445775061623	0.518499471285687\\
0.274975434022527	0.516867045940764\\
0.275491484118885	0.515232852916318\\
0.276005899142151	0.513596901264886\\
0.276518677338225	0.511959200043273\\
0.277029816962067	0.5103197583125\\
0.277539316277701	0.508678585137761\\
0.278047173558218	0.507035689588374\\
0.278553387085778	0.505391080737729\\
0.279057955151619	0.503744767663252\\
0.279560876056057	0.502096759446343\\
0.280062148108492	0.500447065172342\\
0.280561769627412	0.49879569393047\\
0.281059738940395	0.497142654813793\\
0.281556054384114	0.495487956919165\\
0.282050714304341	0.493831609347186\\
0.28254371705595	0.492173621202151\\
0.283035061002923	0.490514001592009\\
0.28352474451835	0.488852759628309\\
0.284012765984434	0.487189904426154\\
0.284499123792496	0.485525445104156\\
0.284983816342976	0.483859390784387\\
0.28546684204544	0.482191750592334\\
0.28594819931858	0.480522533656847\\
0.286427886590217	0.478851749110096\\
0.286905902297308	0.477179406087522\\
0.287382244885946	0.47550551372779\\
0.287856912811365	0.47383008117274\\
0.288329904537941	0.472153117567344\\
0.288801218539199	0.470474632059653\\
0.28927085329781	0.468794633800756\\
0.289738807305601	0.467113131944726\\
0.290205079063553	0.465430135648578\\
0.290669667081805	0.463745654072222\\
0.291132569879658	0.462059696378409\\
0.291593785985578	0.460372271732692\\
0.292053313937196	0.458683389303375\\
0.292511152281314	0.456993058261465\\
0.292967299573906	0.455301287780627\\
0.293421754380121	0.453608087037134\\
0.293874515274285	0.451913465209823\\
0.294325580839906	0.450217431480046\\
0.294774949669673	0.448519995031624\\
0.295222620365458	0.446821165050798\\
0.295668591538326	0.445120950726182\\
0.296112861808527	0.443419361248718\\
0.296555429805504	0.44171640581163\\
0.296996294167895	0.440012093610371\\
0.297435453543535	0.43830643384258\\
0.297872906589458	0.436599435708037\\
0.298308651971897	0.434891108408611\\
0.298742688366288	0.433181461148217\\
0.299175014457275	0.431470503132766\\
0.299605628938705	0.429758243570119\\
0.300034530513636	0.428044691670043\\
0.300461717894336	0.426329856644158\\
0.300887189802287	0.424613747705896\\
0.301310944968182	0.422896374070449\\
0.301732982131934	0.421177744954728\\
0.302153300042671	0.419457869577308\\
0.302571897458741	0.417736757158389\\
0.302988773147714	0.416014416919745\\
0.303403925886382	0.414290858084676\\
0.30381735446076	0.412566089877966\\
0.30422905766609	0.41084012152583\\
0.304639034306839	0.409112962255873\\
0.305047283196705	0.407384621297039\\
0.305453803158613	0.405655107879566\\
0.305858593024722	0.403924431234938\\
0.306261651636419	0.402192600595841\\
0.306662977844327	0.400459625196112\\
0.307062570508302	0.398725514270696\\
0.307460428497438	0.396990277055598\\
0.307856550690063	0.395253922787837\\
0.308250935973742	0.393516460705394\\
0.308643583245281	0.391777900047177\\
0.309034491410724	0.390038250052961\\
0.309423659385356	0.388297519963351\\
0.3098110860937	0.386555719019731\\
0.310196770469524	0.384812856464218\\
0.310580711455838	0.383068941539616\\
0.310962908004895	0.38132398348937\\
0.31134335907819	0.379577991557518\\
0.311722063646464	0.377830974988646\\
0.312099020689704	0.376082943027839\\
0.312474229197138	0.374333904920639\\
0.312847688167245	0.372583869912993\\
0.313219396607746	0.370832847251212\\
0.313589353535611	0.369080846181921\\
0.313957557977056	0.367327875952012\\
0.314324008967544	0.365573945808602\\
0.314688705551785	0.363819064998983\\
0.315051646783736	0.362063242770576\\
0.315412831726602	0.360306488370886\\
0.315772259452836	0.358548811047455\\
0.316129929044138	0.356790220047815\\
0.316485839591455	0.355030724619444\\
0.316839990194983	0.353270334009718\\
0.317192379964164	0.351509057465865\\
0.317543008017686	0.349746904234919\\
0.317891873483487	0.347983883563673\\
0.318238975498749	0.346220004698637\\
0.318584313209901	0.344455276885986\\
0.31892788577262	0.342689709371517\\
0.319269692351826	0.340923311400604\\
0.319609732121684	0.339156092218149\\
0.319948004265606	0.33738806106854\\
0.320284507976247	0.335619227195601\\
0.320619242455504	0.333849599842548\\
0.320952206914519	0.332079188251944\\
0.321283400573676	0.330308001665651\\
0.321612822662599	0.328536049324785\\
0.321940472420154	0.326763340469672\\
0.322266349094447	0.324989884339799\\
0.322590451942822	0.323215690173771\\
0.322912780231862	0.321440767209263\\
0.323233333237386	0.319665124682976\\
0.323552110244452	0.317888771830592\\
0.32386911054735	0.316111717886724\\
0.324184333449606	0.314333972084876\\
0.324497778263978	0.312555543657394\\
0.324809444312456	0.310776441835422\\
0.325119330926261	0.308996675848853\\
0.325427437445842	0.30721625492629\\
0.325733763220877	0.305435188294994\\
0.326038307610271	0.303653485180844\\
0.326341069982153	0.301871154808284\\
0.326642049713876	0.300088206400287\\
0.326941246192016	0.298304649178303\\
0.327238658812368	0.296520492362218\\
0.327534286979948	0.294735745170302\\
0.327828130108989	0.292950416819173\\
0.328120187622938	0.291164516523743\\
0.328410458954459	0.289378053497179\\
0.328698943545426	0.287591036950855\\
0.328985640846924	0.285803476094309\\
0.329270550319247	0.284015380135192\\
0.329553671431896	0.282226758279231\\
0.329835003663575	0.280437619730178\\
0.330114546502192	0.278647973689769\\
0.330392299444857	0.276857829357675\\
0.330668261997876	0.275067195931462\\
0.330942433676754	0.273276082606539\\
0.331214814006188	0.271484498576121\\
0.331485402520068	0.269692453031179\\
0.331754198761475	0.267899955160398\\
0.332021202282675	0.266107014150129\\
0.332286412645123	0.264313639184346\\
0.332549829419453	0.262519839444602\\
0.332811452185481	0.260725624109987\\
0.3330712805322	0.258931002357073\\
0.33332931405778	0.257135983359884\\
0.333585552369562	0.255340576289838\\
0.333839995084058	0.253544790315712\\
0.334092641826948	0.251748634603591\\
0.334343492233074	0.24995211831683\\
0.334592545946443	0.248155250616001\\
0.334839802620221	0.246358040658859\\
0.335085261916727	0.244560497600286\\
0.335328923507438	0.242762630592259\\
0.335570787072978	0.240964448783796\\
0.335810852303121	0.239165961320914\\
0.336049118896783	0.23736717734659\\
0.336285586562024	0.23556810600071\\
0.336520255016039	0.233768756420028\\
0.336753123985163	0.231969137738122\\
0.336984193204859	0.230169259085349\\
0.33721346241972	0.228369129588804\\
0.337440931383463	0.226568758372269\\
0.337666599858931	0.224768154556177\\
0.337890467618081	0.222967327257565\\
0.33811253444199	0.221166285590026\\
0.338332800120842	0.219365038663674\\
0.338551264453933	0.21756359558509\\
0.338767927249663	0.215761965457287\\
0.338982788325531	0.213960157379661\\
0.339195847508137	0.21215818044795\\
0.339407104633173	0.210356043754188\\
0.339616559545421	0.208553756386665\\
0.33982421209875	0.206751327429878\\
0.340030062156111	0.204948765964493\\
0.340234109589535	0.203146081067298\\
0.340436354280126	0.201343281811162\\
0.340636796118061	0.199540377264989\\
0.340835435002581	0.197737376493677\\
0.341032270841993	0.195934288558072\\
0.341227303553661	0.19413112251493\\
0.341420533064005	0.192327887416866\\
0.341611959308493	0.190524592312317\\
0.341801582231642	0.188721246245499\\
0.34198940178701	0.186917858256356\\
0.342175417937193	0.185114437380529\\
0.342359630653819	0.183310992649303\\
0.342542039917548	0.181507533089568\\
0.342722645718063	0.179704067723778\\
0.342901448054067	0.177900605569903\\
0.343078446933278	0.176097155641392\\
0.343253642372428	0.174293726947124\\
0.343427034397252	0.172490328491371\\
0.34359862304249	0.170686969273753\\
0.343768408351877	0.168883658289194\\
0.343936390378143	0.16708040452788\\
0.344102569183003	0.16527721697522\\
0.344266944837156	0.163474104611795\\
0.344429517420281	0.161671076413327\\
0.344590287021028	0.159868141350628\\
0.344749253737016	0.158065308389558\\
0.344906417674828	0.156262586490988\\
0.345061778950004	0.154459984610754\\
0.345215337687039	0.152657511699613\\
0.345367094019376	0.150855176703207\\
0.345517048089398	0.149052988562011\\
0.345665200048429	0.147250956211304\\
0.345811550056727	0.145449088581114\\
0.345956098283474	0.143647394596184\\
0.346098844906775	0.141845883175929\\
0.346239790113652	0.140044563234389\\
0.346378934100039	0.138243443680194\\
0.346516277070776	0.13644253341652\\
0.346651819239602	0.134641841341042\\
0.346785560829151	0.132841376345902\\
0.346917502070949	0.131041147317657\\
0.347047643205403	0.129241163137246\\
0.347175984481798	0.127441432679942\\
0.347302526158294	0.125641964815316\\
0.347427268501914	0.12384276840719\\
0.347550211788546	0.1220438523136\\
0.347671356302931	0.120245225386751\\
0.347790702338658	0.11844689647298\\
0.347908250198161	0.11664887441271\\
0.348024000192712	0.114851168040413\\
0.348137952642414	0.113053786184565\\
0.348250107876194	0.111256737667608\\
0.348360466231801	0.109460031305907\\
0.348469028055795	0.10766367590971\\
0.348575793703544	0.105867680283105\\
0.348680763539217	0.104072053223984\\
0.348783937935778	0.102276803523995\\
0.34888531727498	0.100481939968507\\
0.348984901947357	0.0986874713365675\\
0.349082692352218	0.0968934064008607\\
0.349178688897646	0.0950997539276669\\
0.349272892000481	0.0933065226768244\\
0.349365302086325	0.0915137214016848\\
0.349455919589525	0.0897213588490771\\
0.349544744953177	0.087929443759263\\
0.34963177862911	0.0861379848659\\
0.349717021077883	0.084346990895998\\
0.349800472768783	0.0825564705698813\\
0.349882134179809	0.0807664326011475\\
0.349962005797674	0.0789768856966275\\
0.350040088117791	0.0771878385563434\\
0.350116381644272	0.0753992998734731\\
0.350190886889919	0.073611278334305\\
0.350263604376215	0.0718237826182019\\
0.35033453463332	0.0700368213975579\\
0.350403678200064	0.068250403337762\\
0.350471035623936	0.0664645370971548\\
0.350536607461082	0.0646792313269924\\
0.350600394276295	0.0628944946714015\\
0.350662396643008	0.0611103357673469\\
0.35072261514329	0.0593267632445851\\
0.350781050367832	0.0575437857256288\\
0.350837702915948	0.055761411825706\\
0.350892573395559	0.0539796501527205\\
0.350945662423196	0.0521985093072134\\
0.350996970623981	0.0504179978823231\\
0.351046498631629	0.0486381244637457\\
0.351094247088437	0.0468588976296975\\
0.351140216645274	0.0450803259508731\\
0.351184407961578	0.0433024179904087\\
0.351226821705346	0.0415251823038423\\
0.351267458553126	0.039748627439075\\
0.351306319190011	0.0379727619363303\\
0.351343404309631	0.0361975943281189\\
0.351378714614141	0.034423133139195\\
0.351412250814221	0.0326493868865233\\
0.351444013629063	0.0308763640792344\\
0.351474003786362	0.0291040732185917\\
0.351502222022313	0.0273325227979482\\
0.3515286690816	0.0255617213027113\\
0.351553345717386	0.0237916772103021\\
0.351576252691311	0.0220223989901188\\
0.351597390773478	0.0202538951034966\\
0.351616760742447	0.0184861740036718\\
0.351634363385229	0.0167192441357406\\
0.351650199497275	0.014953113936624\\
0.351664269882468	0.0131877918350269\\
0.351676575353118	0.0114232862514024\\
0.351687116729948	0.00965960559791228\\
0.351695894842092	0.00789675827839119\\
0.351702910527081	0.00613475268830383\\
0.351708164630841	0.00437359721471521\\
0.351711658007676	0.00261330023624368\\
0.351713391520268	0.000853870123031927\\
}
 [postaction={decorate, decoration={markings,
         mark=between positions 0.5 and 1 step 1 with {\arrow[blue,line width=1.5pt]{latex};}
       }}]
;
\addlegendentry{$u = 0$}

\addplot [color=blue, dotted, line width=2.0pt]
  table[row sep=crcr]{%
0.351713391520268	0.000853870123031927\\
0.351713391520268	0.000853870123031927\\
};
%\addlegendentry{data4}

\addplot [color=black, line width=2.0pt]
  table[row sep=crcr]{%
0.351713391520268	0.000853870123031927\\
0.351760697498142	0.00570195204196583\\
0.351874948920672	0.0102141816599909\\
0.35205151148703	0.0144128319411243\\
0.352286033233259	0.0183188139602165\\
0.353352231029928	0.0289105527720193\\
0.354844795624168	0.0373132589722205\\
0.356679595459083	0.0439300232776984\\
0.358779242816386	0.0491290990181305\\
0.361115524620325	0.0532600910935098\\
0.36361815819057	0.0564346264812784\\
0.366249865866127	0.0588271657635912\\
0.368976195027982	0.0605975482317459\\
0.372309315443092	0.062074534715418\\
0.375707260239703	0.0629847672846849\\
0.379145014876719	0.0634392649252497\\
0.382597978542932	0.063546280847004\\
0.386655218055554	0.0633430298312276\\
0.390690015250083	0.0628411265951341\\
0.394687928850088	0.0620976926370327\\
0.398632829898123	0.0611778413001183\\
0.403261064860015	0.0599197939364063\\
0.407787376500753	0.0585056681141061\\
0.412204232956827	0.0569601164342316\\
0.416500047784615	0.0553280189216882\\
0.421386858232834	0.0533510763597701\\
0.426092939370141	0.051302348432883\\
0.430617075996755	0.0491879003542718\\
0.434950727526958	0.0470501488881581\\
0.439567298973511	0.0446820392009227\\
0.443945678062565	0.0423110421686926\\
0.448092057512114	0.0399330546598628\\
0.452002046961233	0.0375942947057486\\
0.455925119791662	0.0351851214125975\\
0.459591874214692	0.0328345271505856\\
0.463015251702851	0.0305325567989189\\
0.466196415876149	0.0283220068977278\\
0.469325240039577	0.0261114968139639\\
0.472206089720713	0.0240050952771054\\
0.474856997838893	0.0219874428989193\\
0.477284029545739	0.0200933005303077\\
0.479666345284601	0.0182195135169922\\
0.481823747637567	0.0164752103814626\\
0.483777682415405	0.0148394531263104\\
0.485537515852561	0.0133387764446514\\
0.487117102950747	0.0119908722254656\\
0.488535669702019	0.0107566737040124\\
0.489809853041184	0.00962043950254697\\
0.490949473347034	0.00859126487830733\\
0.491964156984808	0.00767528224655973\\
0.492870010536195	0.00684683526338842\\
0.493678811754626	0.00609476004475129\\
0.49439880434949	0.0054196214016131\\
0.495037743420087	0.00482085354797967\\
0.495605808645072	0.00428396191554535\\
0.496110890244435	0.00380137441308626\\
0.496559092004293	0.00337079861607639\\
0.496681253027227	0.00325356961059423\\
0.496799162261047	0.00314024230454304\\
0.49691296158403	0.00303069973476688\\
0.497022788413045	0.00292482868251474\\
}
[postaction={decorate, decoration={markings,
        mark=between positions 0.7 and 1 step 1 with {\arrow[black,line width=1.5pt]{latex};}
      }}]
;
%\addlegendentry{data5}

\addplot [color=blue, dotted, line width=2.0pt]
  table[row sep=crcr]{%
0.497022788413045	0.00292482868251474\\
0.497022788413045	0.002988519060116\\
};
%\addlegendentry{data6}

\addplot [color=blue, line width=2.0pt]
  table[row sep=crcr]{%
0.497022788413045	0.002988519060116\\
0.497024533833223	0.000502527008580935\\
0.497023793880453	-0.00198222486963153\\
0.497020569800809	-0.00446572477380426\\
0.49701486285216	-0.00694796091537732\\
0.497006674304158	-0.0094289215180008\\
0.496996005438225	-0.0119085948175876\\
0.496982857547541	-0.0143869690623661\\
0.496967231937032	-0.0168640325129327\\
0.496949129923356	-0.0193397734423045\\
0.496928552834893	-0.0218141801359714\\
0.49690550201173	-0.0242872408919486\\
0.496879978805651	-0.0267589440208291\\
0.496851984580123	-0.0292292778458353\\
0.496821520710282	-0.0316982307028714\\
0.496788588582924	-0.0341657909405754\\
0.496753189596487	-0.0366319469203707\\
0.496715325161045	-0.0390966870165182\\
0.496674996698287	-0.041559999616168\\
0.496632205641511	-0.0440218731194104\\
0.496586953435609	-0.0464822959393284\\
0.496539241537052	-0.0489412565020484\\
0.496489071413878	-0.0513987432467916\\
0.49643644454568	-0.0538547446259257\\
0.496381362423593	-0.0563092491050155\\
0.496323826550277	-0.0587622451628742\\
0.49626383843991	-0.0612137212916146\\
0.496201399618169	-0.0636636659966994\\
0.496136511622219	-0.0661120677969928\\
0.496069176000699	-0.0685589152248102\\
0.495999394313711	-0.0710041968259697\\
0.495927168132803	-0.0734479011598421\\
0.495852499040957	-0.0758900167994015\\
0.495775388632574	-0.0783305323312755\\
0.495695838513463	-0.0807694363557954\\
0.495613850300827	-0.0832067174870467\\
0.495529425623244	-0.0856423643529187\\
0.495442566120662	-0.0880763655951547\\
0.495353273444377	-0.0905087098694015\\
0.495261549257023	-0.0929393858452598\\
0.495167395232558	-0.0953683822063332\\
0.49507081305625	-0.0977956876502782\\
0.49497180442466	-0.100221290888854\\
0.494870371045632	-0.10264518064797\\
0.494766514638277	-0.105067345667737\\
0.494660236932958	-0.107487774702518\\
0.494551539671277	-0.109906456520973\\
0.494440424606061	-0.112323379906109\\
0.494326893501344	-0.114738533655333\\
0.494210948132358	-0.117151906580497\\
0.494092590285516	-0.119563487507946\\
0.493971821758395	-0.121973265278569\\
0.493848644359726	-0.124381228747848\\
0.493723059909376	-0.126787366785904\\
0.493595070238336	-0.129191668277546\\
0.493464677188704	-0.131594122122319\\
0.493331882613669	-0.133994717234556\\
0.493196688377503	-0.13639344254342\\
0.493059096355536	-0.138790286992954\\
0.49291910843415	-0.141185239542133\\
0.492776726510761	-0.143578289164904\\
0.4926319524938	-0.145969424850242\\
0.492484788302705	-0.14835863560219\\
0.4923352358679	-0.150745910439912\\
0.492183297130785	-0.153131238397738\\
0.492028974043715	-0.15551460852521\\
0.49187226856999	-0.157896009887133\\
0.491713182683836	-0.160275431563616\\
0.491551718370393	-0.162652862650126\\
0.491387877625696	-0.165028292257531\\
0.491221662456662	-0.167401709512144\\
0.491053074881075	-0.169773103555775\\
0.490882116927569	-0.172142463545777\\
0.490708790635611	-0.174509778655087\\
0.49053309805549	-0.176875038072279\\
0.490355041248296	-0.179238231001606\\
0.490174622285909	-0.181599346663046\\
0.489991843250981	-0.183958374292352\\
0.489806706236918	-0.186315303141095\\
0.48961921334787	-0.188670122476707\\
0.489429366698708	-0.191022821582534\\
0.489237168415015	-0.193373389757875\\
0.489042620633064	-0.195721816318031\\
0.488845725499807	-0.198068090594349\\
0.488646485172855	-0.200412201934268\\
0.488444901820463	-0.202754139701363\\
0.488240977621516	-0.205093893275392\\
0.488034714765511	-0.20743145205234\\
0.48782611545254	-0.209766805444463\\
0.487615181893275	-0.212099942880336\\
0.48740191630895	-0.214430853804892\\
0.487186320931348	-0.216759527679473\\
0.486968398002782	-0.219085953981869\\
0.486748149776077	-0.221410122206368\\
0.486525578514559	-0.223732021863793\\
0.48630068649203	-0.226051642481554\\
0.486073475992761	-0.228368973603686\\
0.485843949311468	-0.230684004790896\\
0.485612108753298	-0.232996725620608\\
0.485377956633814	-0.235307125687\\
0.485141495278975	-0.237615194601058\\
0.484902727025121	-0.239920921990611\\
0.484661654218954	-0.242224297500377\\
0.484418279217526	-0.24452531079201\\
0.484172604388217	-0.246823951544137\\
0.483924632108719	-0.249120209452405\\
0.483674364767023	-0.251414074229523\\
0.483421804761394	-0.253705535605305\\
0.483166954500364	-0.255994583326714\\
0.482909816402704	-0.258281207157903\\
0.482650392897418	-0.260565396880255\\
0.482388686423714	-0.262847142292434\\
0.482124699430998	-0.265126433210417\\
0.481858434378847	-0.267403259467544\\
0.481589893737	-0.269677610914556\\
0.481319079985332	-0.271949477419639\\
0.481045995613846	-0.274218848868466\\
0.480770643122647	-0.276485715164235\\
0.480493025021928	-0.278750066227717\\
0.480213143831956	-0.281011891997293\\
0.479931002083047	-0.283271182428997\\
0.479646602315553	-0.285527927496557\\
0.479359947079845	-0.287782117191437\\
0.479071038936291	-0.290033741522878\\
0.478779880455244	-0.292282790517937\\
0.47848647421702	-0.294529254221532\\
0.478190822811878	-0.296773122696478\\
0.477892928840009	-0.299014386023534\\
0.477592794911514	-0.301253034301435\\
0.477290423646384	-0.303489057646941\\
0.476985817674486	-0.305722446194873\\
0.476678979635543	-0.307953190098153\\
0.476369912179115	-0.310181279527846\\
0.476058617964582	-0.3124067046732\\
0.475745099661127	-0.314629455741683\\
0.475429359947716	-0.316849522959028\\
0.475111401513079	-0.319066896569267\\
0.474791227055693	-0.321281566834775\\
0.474468839283764	-0.323493524036308\\
0.474144240915208	-0.325702758473042\\
0.473817434677634	-0.327909260462613\\
0.47348842330832	-0.330113020341155\\
0.473157209554204	-0.332314028463339\\
0.472823796171855	-0.334512275202416\\
0.472488185927463	-0.33670775095025\\
0.472150381596815	-0.33890044611736\\
0.471810385965279	-0.341090351132958\\
0.471468201827784	-0.343277456444987\\
0.471123831988801	-0.34546175252016\\
0.470777279262327	-0.347643229844\\
0.470428546471863	-0.349821878920874\\
0.470077636450394	-0.351997690274033\\
0.469724552040375	-0.354170654445653\\
0.469369296093709	-0.356340761996866\\
0.469011871471728	-0.358508003507807\\
0.468652281045175	-0.360672369577641\\
0.468290527694181	-0.36283385082461\\
0.467926614308254	-0.364992437886064\\
0.467560543786252	-0.367148121418501\\
0.467192319036368	-0.369300892097603\\
0.46682194297611	-0.371450740618276\\
0.46644941853228	-0.373597657694681\\
0.466074748640958	-0.375741634060276\\
0.465697936247479	-0.377882660467852\\
0.465318984306417	-0.380020727689568\\
0.464937895781565	-0.382155826516986\\
0.464554673645912	-0.384287947761112\\
0.464169320881627	-0.386417082252428\\
0.463781840480041	-0.38854322084093\\
0.463392235441623	-0.390666354396165\\
0.463000508775964	-0.392786473807262\\
0.462606663501754	-0.394903569982975\\
0.462210702646767	-0.397017633851713\\
0.461812629247837	-0.399128656361577\\
0.461412446350841	-0.401236628480397\\
0.461010157010678	-0.403341541195767\\
0.460605764291248	-0.405443385515075\\
0.460199271265437	-0.407542152465548\\
0.459790681015092	-0.409637833094276\\
0.459379996631001	-0.411730418468257\\
0.458967221212878	-0.413819899674422\\
0.458552357869339	-0.415906267819679\\
0.458135409717883	-0.417989514030938\\
0.457716379884873	-0.420069629455155\\
0.457295271505513	-0.422146605259357\\
0.456872087723832	-0.424220432630685\\
0.456446831692661	-0.42629110277642\\
0.456019506573612	-0.428358606924021\\
0.455590115537063	-0.430422936321161\\
0.455158661762131	-0.432484082235754\\
0.454725148436657	-0.434542035955994\\
0.454289578757181	-0.436596788790388\\
0.453851955928928	-0.438648332067787\\
0.453412283165782	-0.440696657137419\\
0.452970563690266	-0.442741755368926\\
0.452526800733527	-0.444783618152392\\
0.452080997535308	-0.446822236898379\\
0.451633157343932	-0.448857603037959\\
0.451183283416284	-0.450889708022745\\
0.450731379017781	-0.452918543324927\\
0.450277447422364	-0.454944100437301\\
0.449821491912466	-0.456966370873302\\
0.449363515778998	-0.458985346167036\\
0.448903522321328	-0.461001017873315\\
0.448441514847256	-0.463013377567684\\
0.447977496672998	-0.465022416846456\\
0.447511471123163	-0.467028127326741\\
0.447043441530731	-0.469030500646483\\
0.446573411237037	-0.471029528464483\\
0.446101383591743	-0.473025202460439\\
0.445627361952824	-0.475017514334971\\
0.44515134968654	-0.477006455809653\\
0.444673350167424	-0.478992018627048\\
0.444193366778251	-0.480974194550732\\
0.443711402910025	-0.482952975365332\\
0.443227461961955	-0.484928352876551\\
0.442741547341432	-0.4869003189112\\
0.442253662464009	-0.48886886531723\\
0.441763810753384	-0.49083398396376\\
0.441271995641372	-0.492795666741109\\
0.440778220567888	-0.494753905560823\\
0.440282488980925	-0.496708692355709\\
0.439784804336533	-0.498660019079859\\
0.439285170098797	-0.500607877708688\\
0.438783589739816	-0.502552260238953\\
0.43828006673968	-0.504493158688791\\
0.437774604586452	-0.506430565097745\\
0.437267206776145	-0.508364471526792\\
0.436757876812698	-0.510294870058374\\
0.436246618207959	-0.512221752796424\\
0.435733434481661	-0.514145111866399\\
0.4352183291614	-0.516064939415304\\
0.434701305782615	-0.517981227611725\\
0.434182367888565	-0.519893968645853\\
0.433661519030307	-0.521803154729516\\
0.433138762766678	-0.523708778096204\\
0.432614102664269	-0.525610831001099\\
0.432087542297404	-0.527509305721102\\
0.431559085248122	-0.529404194554862\\
0.43102873510615	-0.531295489822801\\
0.430496495468885	-0.533183183867145\\
0.429962369941369	-0.535067269051949\\
0.429426362136272	-0.536947737763123\\
0.428888475673865	-0.538824582408463\\
0.428348714182001	-0.540697795417676\\
0.427807081296092	-0.542567369242406\\
0.427263580659087	-0.544433296356261\\
0.426718215921451	-0.546295569254842\\
0.426170990741143	-0.548154180455766\\
0.425621908783593	-0.550009122498695\\
0.425070973721678	-0.551860387945361\\
0.424518189235706	-0.553707969379594\\
0.423963559013388	-0.555551859407346\\
0.423407086749818	-0.557392050656715\\
0.422848776147451	-0.559228535777977\\
0.422288630916081	-0.561061307443606\\
0.421726654772818	-0.562890358348301\\
0.421162851442066	-0.564715681209013\\
0.420597224655501	-0.56653726876497\\
0.420029778152051	-0.568355113777699\\
0.419460515677866	-0.570169209031055\\
0.418889440986306	-0.571979547331243\\
0.418316557837912	-0.573786121506845\\
0.417741870000384	-0.575588924408845\\
0.417165381248561	-0.57738794891065\\
0.416587095364398	-0.579183187908117\\
0.416007016136941	-0.580974634319577\\
0.415425147362307	-0.582762281085862\\
0.414841492843663	-0.584546121170322\\
0.414256056391199	-0.586326147558854\\
0.413668841822108	-0.588102353259928\\
0.413079852960565	-0.589874731304604\\
0.4124890936377	-0.591643274746561\\
0.411896567691581	-0.593407976662116\\
0.411302278967185	-0.595168830150254\\
0.410706231316382	-0.596925828332644\\
0.410108428597907	-0.598678964353666\\
0.40950887467734	-0.600428231380432\\
0.408907573427081	-0.60217362260281\\
0.408304528726332	-0.603915131233448\\
0.407699744461067	-0.605652750507792\\
0.407093224524016	-0.607386473684113\\
0.406484972814638	-0.60911629404353\\
0.4058749932391	-0.610842204890024\\
0.405263289710253	-0.612564199550472\\
0.404649866147611	-0.61428227137466\\
0.404034726477326	-0.615996413735305\\
0.403417874632164	-0.617706620028084\\
0.402799314551486	-0.619412883671647\\
0.402179050181223	-0.621115198107645\\
0.401557085473852	-0.622813556800747\\
0.400933424388374	-0.624507953238662\\
0.40030807089029	-0.626198380932161\\
0.39968102895158	-0.6278848334151\\
0.399052302550678	-0.629567304244434\\
0.398421895672449	-0.631245787000245\\
0.397789812308168	-0.632920275285759\\
0.397156056455492	-0.634590762727367\\
0.396520632118445	-0.636257242974645\\
0.395883543307384	-0.637919709700374\\
0.395244794038986	-0.639578156600561\\
0.394604388336218	-0.641232577394457\\
0.393962330228319	-0.64288296582458\\
0.393318623750769	-0.644529315656732\\
0.392673272945275	-0.646171620680019\\
0.392026281859741	-0.647809874706871\\
0.391377654548248	-0.64944407157306\\
0.390727395071029	-0.651074205137721\\
0.390075507494446	-0.652700269283369\\
0.389421995890967	-0.654322257915922\\
0.388766864339142	-0.655940164964713\\
0.388110116923582	-0.657553984382514\\
0.387451757734929	-0.659163710145554\\
0.386791790869843	-0.660769336253533\\
0.386130220430966	-0.662370856729647\\
0.385467050526912	-0.663968265620602\\
0.38480228527223	-0.66556155699663\\
0.384135928787393	-0.667150724951513\\
0.383467985198762	-0.668735763602597\\
0.382798458638575	-0.670316667090807\\
0.382127353244914	-0.671893429580669\\
0.381454673161684	-0.673466045260327\\
0.380780422538593	-0.675034508341558\\
0.380104605531122	-0.676598813059791\\
0.379427226300508	-0.678158953674122\\
0.378748289013715	-0.679714924467332\\
0.378067797843412	-0.681266719745905\\
0.377385756967951	-0.682814333840042\\
0.376702170571342	-0.684357761103681\\
0.376017042843229	-0.685896995914508\\
0.375330377978864	-0.68743203267398\\
0.37464218017909	-0.688962865807335\\
0.373952453650309	-0.690489489763611\\
0.373261202604464	-0.692011899015663\\
0.372568431259013	-0.693530088060175\\
0.371874143836906	-0.695044051417679\\
0.371178344566559	-0.696553783632569\\
0.370481037681833	-0.698059279273117\\
0.369782227422007	-0.699560532931486\\
0.369081918031759	-0.701057539223749\\
0.368380113761135	-0.7025502927899\\
0.367676818865533	-0.704038788293873\\
0.366972037605673	-0.70552302042355\\
0.366265774247574	-0.707002983890785\\
0.365558033062534	-0.708478673431411\\
0.3648488183271	-0.709950083805257\\
0.364138134323051	-0.71141720979616\\
0.363425985337367	-0.712880046211984\\
0.362712375662208	-0.714338587884629\\
0.361997309594893	-0.715792829670049\\
0.361280791437871	-0.71724276644826\\
0.360562825498698	-0.71868839312336\\
0.359843416090016	-0.720129704623536\\
0.359122567529526	-0.721566695901087\\
0.358400284139963	-0.722999361932423\\
0.357676570249076	-0.724427697718092\\
0.356951430189601	-0.725851698282785\\
0.356224868299235	-0.727271358675349\\
0.355496888920617	-0.728686673968804\\
0.354767496401298	-0.730097639260352\\
0.354036695093723	-0.73150424967139\\
0.353304489355201	-0.732906500347522\\
0.352570883547883	-0.734304386458574\\
0.351835882038741	-0.735697903198601\\
0.351099489199537	-0.737087045785904\\
0.350361709406806	-0.73847180946304\\
0.349622547041826	-0.73985218949683\\
0.348882006490597	-0.741228181178376\\
0.348140092143815	-0.74259977982307\\
0.34739680839685	-0.743966980770607\\
0.346652159649718	-0.74532977938499\\
0.345906150307062	-0.74668817105455\\
0.34515878477812	-0.748042151191951\\
0.34441006747671	-0.749391715234202\\
0.343660002821197	-0.750736858642666\\
0.342908595234474	-0.752077576903077\\
0.342155849143937	-0.753413865525541\\
0.341401768981458	-0.754745720044554\\
0.340646359183363	-0.756073136019009\\
0.339889624190407	-0.757396109032204\\
0.33913156844775	-0.758714634691857\\
0.338372196404931	-0.76002870863011\\
0.337611512515845	-0.761338326503543\\
0.336849521238719	-0.762643483993182\\
0.336086227036085	-0.763944176804507\\
0.335321634374759	-0.765240400667465\\
0.334555747725815	-0.766532151336474\\
0.333788571564558	-0.767819424590435\\
0.333020110370505	-0.769102216232743\\
0.332250368627354	-0.770380522091289\\
0.331479350822967	-0.771654338018478\\
0.330707061449336	-0.772923659891227\\
0.329933505002569	-0.774188483610983\\
0.329158685982857	-0.775448805103725\\
0.328382608894454	-0.776704620319973\\
0.327605278245651	-0.777955925234799\\
0.326826698548752	-0.779202715847832\\
0.326046874320048	-0.780444988183265\\
0.325265810079795	-0.781682738289868\\
0.324483510352187	-0.782915962240987\\
0.323699979665332	-0.784144656134559\\
0.32291522255123	-0.785368816093116\\
0.322129243545742	-0.786588438263791\\
0.321342047188574	-0.787803518818326\\
0.320553638023246	-0.789014053953081\\
0.319764020597068	-0.790220039889037\\
0.318973199461119	-0.791421472871805\\
0.318181179170219	-0.792618349171632\\
0.317387964282905	-0.793810665083406\\
0.316593559361409	-0.794998416926665\\
0.315797968971627	-0.796181601045601\\
0.315001197683103	-0.797360213809064\\
0.314203250068997	-0.798534251610574\\
0.313404130706063	-0.799703710868319\\
0.312603844174627	-0.800868588025169\\
0.311802395058558	-0.802028879548672\\
0.310999787945244	-0.803184581931067\\
0.310196027425571	-0.804335691689286\\
0.309391118093893	-0.80548220536496\\
0.308585064548013	-0.806624119524422\\
0.307777871389153	-0.807761430758715\\
0.306969543221932	-0.808894135683593\\
0.306160084654342	-0.810022230939528\\
0.305349500297722	-0.811145713191715\\
0.304537794766731	-0.812264579130076\\
0.303724972679328	-0.813378825469259\\
0.302911038656746	-0.814488448948651\\
0.302095997323463	-0.815593446332376\\
0.301279853307183	-0.816693814409299\\
0.300462611238809	-0.817789549993034\\
0.299644275752416	-0.81888064992194\\
0.298824851485229	-0.819967111059133\\
0.298004343077599	-0.821048930292483\\
0.297182755172975	-0.82212610453462\\
0.296360092417881	-0.823198630722936\\
0.295536359461893	-0.82426650581959\\
0.294711560957611	-0.825329726811506\\
0.293885701560635	-0.826388290710382\\
0.293058785929543	-0.827442194552688\\
0.292230818725862	-0.828491435399669\\
0.291401804614047	-0.829536010337351\\
0.290571748261453	-0.830575916476537\\
0.289740654338315	-0.831611150952815\\
0.288908527517715	-0.832641710926556\\
0.288075372475568	-0.833667593582916\\
0.287241193890586	-0.834688796131842\\
0.286405996444262	-0.835705315808067\\
0.285569784820841	-0.836717149871117\\
0.284732563707296	-0.837724295605309\\
0.283894337793304	-0.838726750319754\\
0.283055111771219	-0.839724511348356\\
0.28221489033605	-0.840717576049817\\
0.281373678185434	-0.841705941807632\\
0.280531480019613	-0.842689606030094\\
0.279688300541408	-0.843668566150294\\
0.278844144456194	-0.84464281962612\\
0.277999016471877	-0.845612363940259\\
0.277152921298865	-0.846577196600193\\
0.276305863650051	-0.847537315138208\\
0.275457848240778	-0.848492717111382\\
0.274608879788822	-0.849443400101597\\
0.273758963014367	-0.850389361715528\\
0.272908102639973	-0.851330599584649\\
0.272056303390561	-0.852267111365232\\
0.271203569993379	-0.853198894738345\\
0.270349907177985	-0.854125947409848\\
0.269495319676216	-0.855048267110399\\
0.268639812222169	-0.855965851595448\\
0.267783389552171	-0.856878698645238\\
0.266926056404756	-0.857786806064801\\
0.266067817520642	-0.858690171683959\\
0.265208677642704	-0.859588793357324\\
0.264348641515949	-0.860482668964292\\
0.263487713887496	-0.861371796409044\\
0.262625899506543	-0.862256173620543\\
0.261763203124348	-0.863135798552533\\
0.260899629494205	-0.864010669183538\\
0.260035183371414	-0.864880783516854\\
0.259169869513261	-0.865746139580554\\
0.258303692678992	-0.866606735427482\\
0.257436657629786	-0.867462569135248\\
0.256568769128734	-0.868313638806231\\
0.255700031940811	-0.86915994256757\\
0.254830450832852	-0.870001478571165\\
0.253960030573529	-0.870838244993672\\
0.253088775933325	-0.871670240036501\\
0.252216691684508	-0.872497461925813\\
0.251343782601108	-0.873319908912513\\
0.250470053458892	-0.874137579272251\\
0.249595509035339	-0.874950471305415\\
0.248720154109615	-0.875758583337129\\
0.247843993462549	-0.876561913717248\\
0.246967031876606	-0.877360460820352\\
0.246089274135868	-0.878154223045748\\
0.245210725026	-0.878943198817459\\
0.244331389334235	-0.879727386584222\\
0.243451271849342	-0.880506784819484\\
0.242570377361607	-0.881281392021396\\
0.241688710662803	-0.882051206712812\\
0.240806276546169	-0.882816227441276\\
0.239923079806384	-0.883576452779026\\
0.239039125239543	-0.884331881322982\\
0.23815441764313	-0.885082511694744\\
0.237268961815996	-0.885828342540588\\
0.236382762558334	-0.886569372531455\\
0.235495824671653	-0.887305600362951\\
0.234608152958754	-0.888037024755338\\
0.233719752223707	-0.888763644453529\\
0.232830627271823	-0.889485458227081\\
0.23194078290963	-0.890202464870192\\
0.231050223944853	-0.890914663201688\\
0.230158955186383	-0.891622052065027\\
0.229266981444256	-0.892324630328281\\
0.228374307529629	-0.893022396884139\\
0.227480938254752	-0.893715350649891\\
0.226586878432946	-0.894403490567433\\
0.225692132878578	-0.895086815603246\\
0.224796706407036	-0.895765324748402\\
0.223900603834706	-0.896439017018546\\
0.223003829978945	-0.897107891453897\\
0.222106389658055	-0.897771947119235\\
0.221208287691267	-0.898431183103896\\
0.220309528898704	-0.899085598521762\\
0.219410118101366	-0.899735192511257\\
0.218510060121103	-0.900379964235337\\
0.217609359780588	-0.901019912881479\\
0.216708021903296	-0.901655037661678\\
0.215806051313475	-0.902285337812435\\
0.214903452836127	-0.902910812594749\\
0.21400023129698	-0.903531461294112\\
0.213096391522463	-0.904147283220495\\
0.212191938339685	-0.904758277708342\\
0.211286876576407	-0.905364444116563\\
0.210381211061018	-0.905965781828519\\
0.209474946622513	-0.906562290252021\\
0.208568088090466	-0.907153968819313\\
0.207660640295008	-0.907740816987067\\
0.206752608066799	-0.908322834236374\\
0.205843996237007	-0.908900020072732\\
0.204934809637281	-0.909472374026036\\
0.204025053099731	-0.91003989565057\\
0.203114731456897	-0.910602584524999\\
0.20220384954173	-0.911160440252352\\
0.201292412187566	-0.911713462460019\\
0.2003804242281	-0.912261650799737\\
0.199467890497364	-0.91280500494758\\
0.198554815829702	-0.91334352460395\\
0.197641205059745	-0.913877209493564\\
0.196727063022388	-0.914406059365444\\
0.195812394552763	-0.914930073992909\\
0.194897204486218	-0.915449253173559\\
0.193981497658291	-0.915963596729267\\
0.193065278904686	-0.916473104506166\\
0.192148553061248	-0.916977776374642\\
0.191231324963941	-0.917477612229317\\
0.190313599448821	-0.917972611989039\\
0.189395381352014	-0.918462775596872\\
0.18847667550969	-0.918948103020084\\
0.187557486758039	-0.919428594250132\\
0.186637819933249	-0.919904249302655\\
0.18571767987148	-0.920375068217457\\
0.184797071408839	-0.920841051058497\\
0.183875999381358	-0.921302197913877\\
0.182954468624969	-0.921758508895827\\
0.182032483975479	-0.922209984140698\\
0.181110050268547	-0.922656623808942\\
0.180187172339659	-0.923098428085103\\
0.179263855024105	-0.923535397177806\\
0.178340103156954	-0.923967531319739\\
0.17741592157303	-0.924394830767644\\
0.17649131510689	-0.924817295802301\\
0.175566288592795	-0.925234926728515\\
0.174640846864693	-0.925647723875107\\
0.173714994756187	-0.926055687594891\\
0.17278873710052	-0.92645881826467\\
0.171862078730542	-0.926857116285216\\
0.170935024478691	-0.927250582081258\\
0.170007579176971	-0.927639216101468\\
0.169079747656923	-0.928023018818447\\
0.168151534749604	-0.928401990728708\\
0.167222945285561	-0.928776132352667\\
0.166293984094813	-0.929145444234624\\
0.165364656006817	-0.929509926942748\\
0.164434965850455	-0.929869581069065\\
0.163504918454001	-0.930224407229444\\
0.162574518645105	-0.930574406063577\\
0.161643771250762	-0.930919578234967\\
0.160712681097294	-0.931259924430914\\
0.159781253010322	-0.931595445362497\\
0.158849491814745	-0.931926141764562\\
0.157917402334717	-0.932252014395701\\
0.156984989393617	-0.932573064038243\\
0.156052257814035	-0.932889291498233\\
0.155119212417738	-0.933200697605416\\
0.154185858025657	-0.933507283213228\\
0.153252199457853	-0.933809049198771\\
0.1523182415335	-0.934105996462801\\
0.15138398907086	-0.934398125929712\\
0.150449446887257	-0.93468543854752\\
0.149514619799058	-0.934967935287843\\
0.148579512621643	-0.935245617145888\\
0.147644130169387	-0.935518485140433\\
0.146708477255636	-0.935786540313809\\
0.145772558692678	-0.936049783731886\\
0.144836379291728	-0.936308216484052\\
0.143899943862895	-0.936561839683198\\
0.142963257215167	-0.936810654465702\\
0.142026324156383	-0.937054661991409\\
0.141089149493209	-0.937293863443613\\
0.140151738031118	-0.937528260029044\\
0.139214094574364	-0.937757852977844\\
0.138276223925957	-0.937982643543554\\
0.137338130887645	-0.938202633003095\\
0.136399820259886	-0.938417822656746\\
0.135461296841826	-0.938628213828133\\
0.134522565431275	-0.938833807864205\\
0.133583630824686	-0.939034606135216\\
0.132644497817129	-0.939230610034711\\
0.131705171202269	-0.9394218209795\\
0.130765655772342	-0.939608240409648\\
0.129825956318134	-0.939789869788449\\
0.128886077628954	-0.93996671060241\\
0.127946024492614	-0.940138764361234\\
0.127005801695405	-0.940306032597794\\
0.126065414022074	-0.940468516868124\\
0.125124866255798	-0.94062621875139\\
0.124184163178166	-0.940779139849876\\
0.123243309569152	-0.940927281788963\\
0.122302310207093	-0.941070646217109\\
0.121361169868665	-0.94120923480583\\
0.120419893328863	-0.941343049249678\\
0.119478485360975	-0.941472091266226\\
0.118536950736559	-0.941596362596042\\
0.117595294225424	-0.941715865002672\\
0.1166535205956	-0.941830600272619\\
0.115711634613321	-0.941940570215323\\
0.114769641043001	-0.94204577666314\\
0.113827544647208	-0.942146221471321\\
0.112885350186646	-0.942241906517993\\
0.111943062420126	-0.942332833704137\\
0.11100068610455	-0.942419004953565\\
0.110058225994883	-0.942500422212904\\
0.109115686844131	-0.942577087451569\\
0.108173073403323	-0.942649002661748\\
0.107230390421479	-0.942716169858375\\
0.106287642645598	-0.942778591079111\\
0.105344834820627	-0.942836268384323\\
0.104401971689441	-0.942889203857061\\
0.103459057992821	-0.942937399603039\\
0.102516098469434	-0.942980857750608\\
0.101573097855803	-0.943019580450741\\
0.100630060886291	-0.943053569877002\\
0.0996869922930753	-0.943082828225534\\
0.0987438968061265	-0.943107357715028\\
0.0978007791531848	-0.943127160586706\\
0.0968576440597379	-0.943142239104298\\
0.0959144962489988	-0.943152595554016\\
0.0949713404418833	-0.943158232244534\\
0.0940281813569862	-0.943159151506965\\
0.0930850237105611	-0.94315535569484\\
0.0921418722164962	-0.94314684718408\\
0.0911987315862934	-0.943133628372977\\
0.0902556065290438	-0.943115701682172\\
0.089312501751408	-0.943093069554625\\
0.0883694219575919	-0.943065734455599\\
0.087426371849325	-0.943033698872635\\
0.0864833561258384	-0.942996965315524\\
0.0855403794838417	-0.942955536316288\\
0.084597446617502	-0.942909414429154\\
0.0836545622184209	-0.942858602230531\\
0.0827117309756121	-0.942803102318987\\
0.0817689575754804	-0.942742917315222\\
0.0808262467017985	-0.942678049862046\\
0.0798836030356846	-0.942608502624355\\
0.0789410312555825	-0.942534278289105\\
0.0779985360372364	-0.942455379565289\\
0.0770561220536714	-0.942371809183913\\
0.0761137939751702	-0.942283569897968\\
0.0751715564692522	-0.94219066448241\\
0.0742294142006501	-0.94209309573413\\
0.0732873718312888	-0.941990866471933\\
0.0723454340202641	-0.941883979536514\\
0.0714036054238196	-0.941772437790426\\
0.0704618906953255	-0.941656244118063\\
0.0695202944852569	-0.941535401425629\\
0.0685788214411719	-0.941409912641116\\
0.0676374762076892	-0.941279780714275\\
0.0666962634264672	-0.941145008616594\\
0.0657551877361821	-0.941005599341271\\
0.0648142537725059	-0.940861555903186\\
0.0638734661680846	-0.940712881338879\\
0.0629328295525177	-0.940559578706521\\
0.0619923485523345	-0.940401651085889\\
0.0610520277909752	-0.940239101578338\\
0.0601118718887661	-0.94007193330678\\
0.0591718854629016	-0.93990014941565\\
0.0582320731274198	-0.939723753070886\\
0.0572924394931822	-0.9395427474599\\
0.0563529891678527	-0.939357135791549\\
0.055413726755875	-0.939166921296113\\
0.0544746568584525	-0.938972107225266\\
0.0535357840735255	-0.938772696852045\\
0.0525971129957511	-0.938568693470829\\
0.0516586482164814	-0.938360100397311\\
0.050720394323742	-0.938146920968467\\
0.0497823559022105	-0.937929158542532\\
0.0488445375331966	-0.937706816498971\\
0.0479069437946185	-0.937479898238452\\
0.0469695792609848	-0.93724840718282\\
0.0460324485033706	-0.937012346775065\\
0.0450955560893975	-0.936771720479299\\
0.044158906583213	-0.936526531780727\\
0.0432225045454684	-0.936276784185614\\
0.0422863545332986	-0.936022481221267\\
0.0413504611003	-0.935763626435996\\
0.0404148287965107	-0.935500223399091\\
0.0394794621683889	-0.935232275700798\\
0.0385443657587928	-0.934959786952281\\
0.0376095441069574	-0.9346827607856\\
0.0366750017484768	-0.934401200853682\\
0.0357407432152807	-0.934115110830289\\
0.0348067730356153	-0.933824494409993\\
0.0338730957340216	-0.933529355308146\\
0.0329397158313145	-0.933229697260848\\
0.0320066378445629	-0.932925524024923\\
0.0310738662870685	-0.932616839377886\\
0.0301414056683449	-0.932303647117916\\
0.0292092604940976	-0.931985951063825\\
0.0282774352662026	-0.93166375505503\\
0.027345934482686	-0.931337062951522\\
0.0264147626377044	-0.931005878633839\\
0.0254839242215228	-0.930670206003033\\
0.0245534237204959	-0.930330048980642\\
0.023623265617045	-0.929985411508661\\
0.0226934543896411	-0.929636297549512\\
0.0217639945127812	-0.92928271108601\\
0.0208348904569697	-0.928924656121339\\
0.0199061466886979	-0.928562136679018\\
0.0189777676704231	-0.92819515680287\\
0.018049757860549	-0.927823720556995\\
0.0171221217134045	-0.927447832025738\\
0.0161948636792247	-0.927067495313656\\
0.0152679882041294	-0.926682714545491\\
0.0143414997301042	-0.926293493866139\\
0.0134154026949794	-0.925899837440615\\
0.0124897015324103	-0.92550174945403\\
0.0115644006718569	-0.925099234111548\\
0.0106395045385642	-0.924692295638371\\
0.00971501755354217	-0.924280938279691\\
0.00879094413354512	-0.923865166300672\\
0.00786728869105287	-0.923444983986411\\
0.00694405563424933	-0.923020395641911\\
0.00602124936700441	-0.922591405592045\\
0.0050988742888525	-0.922158018181531\\
0.00417693479497348	-0.921720237774893\\
0.00325543527617338	-0.921278068756433\\
0.00233438011886353	-0.920831515530202\\
0.00141377370504127	-0.920380582519962\\
0.000493620412270762	-0.919925274169158\\
-0.000426075386337674	-0.919465594940886\\
-0.00134530932214601	-0.919001549317858\\
-0.00226407703100885	-0.918533141802374\\
-0.00318237415329365	-0.918060376916285\\
-0.00410019633389833	-0.917583259200966\\
-0.00501753922227249	-0.917101793217277\\
-0.00593439847243649	-0.916615983545538\\
-0.00685076974300079	-0.916125834785489\\
-0.00776664869718424	-0.915631351556262\\
-0.00868203100283605	-0.915132538496347\\
-0.00959691233245272	-0.91462940026356\\
-0.0105112883631977	-0.914121941535007\\
-0.0114251547769225	-0.913610167007055\\
-0.0123385072601837	-0.913094081395295\\
-0.0132513415042633	-0.912573689434515\\
-0.0141636532051874	-0.912048995878657\\
-0.0150754380637463	-0.911520005500792\\
-0.0159866917855122	-0.910986723093085\\
-0.0168974100808593	-0.910449153466758\\
-0.0178075886649827	-0.90990730145206\\
-0.0187172232579169	-0.909361171898232\\
-0.0196263095845554	-0.908810769673472\\
-0.0205348433746691	-0.908256099664905\\
-0.0214428203629257	-0.907697166778544\\
-0.0223502362889083	-0.90713397593926\\
-0.0232570868971344	-0.906566532090747\\
-0.0241633679370741	-0.905994840195486\\
-0.0250690751631698	-0.905418905234715\\
-0.0259742043348549	-0.904838732208388\\
-0.0268787512165709	-0.904254326135149\\
-0.0277827115777884	-0.90366569205229\\
-0.028686081193024	-0.903072835015722\\
-0.0295888558418599	-0.902475760099935\\
-0.0304910313089615	-0.901874472397971\\
-0.0313926033840973	-0.901268977021381\\
-0.0322935678621559	-0.900659279100195\\
-0.0331939205431654	-0.900045383782887\\
-0.0340936572323117	-0.899427296236339\\
-0.0349927737399566	-0.898805021645804\\
-0.0358912658816564	-0.898178565214877\\
-0.0367891294781801	-0.897547932165452\\
-0.0376863603555277	-0.896913127737692\\
-0.0385829543449485	-0.896274157189992\\
-0.0394789072829594	-0.895631025798946\\
-0.0403742150113634	-0.894983738859305\\
-0.0412688733772661	-0.89433230168395\\
-0.0421628782330962	-0.89367671960385\\
-0.0430562254366217	-0.893016997968028\\
-0.0439489108509691	-0.892353142143529\\
-0.0448409303446407	-0.891685157515378\\
-0.0457322797915327	-0.891013049486546\\
-0.0466229550709537	-0.890336823477918\\
-0.0475129520676415	-0.889656484928254\\
-0.0484022666717818	-0.888972039294149\\
-0.0492908947790261	-0.888283492050005\\
-0.050178832290509	-0.887590848687988\\
-0.0510660751128661	-0.886894114717996\\
-0.0519526191582513	-0.886193295667618\\
-0.0528384603443559	-0.885488397082101\\
-0.0537235945944245	-0.884779424524314\\
-0.0546080178372737	-0.88406638357471\\
-0.0554917260073093	-0.883349279831288\\
-0.0563747150445434	-0.882628118909557\\
-0.0572569808946131	-0.881902906442502\\
-0.0581385195087966	-0.881173648080544\\
-0.0590193268440317	-0.880440349491504\\
-0.0598993988629321	-0.879703016360564\\
-0.0607787315338061	-0.878961654390234\\
-0.0616573208306725	-0.878216269300313\\
-0.062535162733279	-0.877466866827848\\
-0.0634122532271187	-0.876713452727105\\
-0.0642885883034474	-0.875956032769522\\
-0.0651641639593014	-0.87519461274368\\
-0.0660389761975137	-0.874429198455259\\
-0.0669130210267317	-0.873659795727005\\
-0.0677862944614341	-0.87288641039869\\
-0.068658792521948	-0.872109048327073\\
-0.069530511234465	-0.871327715385868\\
-0.0704014466310597	-0.8705424174657\\
-0.0712715947497056	-0.869753160474068\\
-0.072140951634292	-0.86895995033531\\
-0.0730095133346407	-0.868162792990565\\
-0.0738772759065235	-0.867361694397729\\
-0.0747442354116781	-0.866556660531426\\
-0.0756103879178252	-0.865747697382961\\
-0.0764757294986857	-0.864934810960286\\
-0.0773402562339959	-0.864118007287964\\
-0.0782039642095253	-0.863297292407126\\
-0.0790668495170931	-0.862472672375434\\
-0.0799289082545839	-0.861644153267043\\
-0.0807901365259649	-0.860811741172563\\
-0.0816505304413025	-0.859975442199018\\
-0.0825100861167775	-0.85913526246981\\
-0.0833687996747031	-0.85829120812468\\
-0.0842266672435402	-0.857443285319664\\
-0.0850836849579132	-0.856591500227064\\
-0.0859398489586278	-0.855735859035398\\
-0.0867951553926856	-0.854876367949369\\
-0.0876496004133018	-0.854013033189823\\
-0.0885031801799196	-0.85314586099371\\
-0.0893558908582277	-0.852274857614042\\
-0.090207728620176	-0.85140002931986\\
-0.0910586896439908	-0.850521382396189\\
-0.0919087701141919	-0.849638923144002\\
-0.0927579662216081	-0.848752657880177\\
-0.0936062741633927	-0.847862592937462\\
-0.0944536901430394	-0.846968734664431\\
-0.0953002103703994	-0.846071089425449\\
-0.0961458310616948	-0.845169663600627\\
-0.096990548439537	-0.844264463585786\\
-0.09783435873294	-0.843355495792416\\
-0.0986772581773378	-0.842442766647638\\
-0.0995192430145991	-0.841526282594159\\
-0.100360309493042	-0.840606050090238\\
-0.101200453867454	-0.839682075609642\\
-0.102039672399099	-0.838754365641607\\
-0.102877961355742	-0.837822926690798\\
-0.10371531701166	-0.836887765277271\\
-0.104551735647654	-0.835948887936426\\
-0.105387213551073	-0.835006301218975\\
-0.106221747015822	-0.834060011690896\\
-0.107055332342379	-0.833110025933395\\
-0.107887965837811	-0.832156350542864\\
-0.10871964381579	-0.831198992130841\\
-0.109550362596608	-0.83023795732397\\
-0.110380118507187	-0.82927325276396\\
-0.111208907881103	-0.828304885107545\\
-0.112036727058594	-0.82733286102644\\
-0.112863572386577	-0.826357187207305\\
-0.113689440218663	-0.8253778703517\\
-0.114514326915173	-0.824394917176047\\
-0.115338228843152	-0.823408334411585\\
-0.116161142376382	-0.822418128804336\\
-0.1169830638954	-0.821424307115055\\
-0.11780398978751	-0.820426876119196\\
-0.118623916446799	-0.819425842606868\\
-0.119442840274153	-0.818421213382794\\
-0.120260757677267	-0.817412995266267\\
-0.121077665070666	-0.816401195091115\\
-0.121893558875712	-0.815385819705652\\
-0.122708435520627	-0.814366875972644\\
-0.123522291440501	-0.81334437076926\\
-0.124335123077306	-0.812318310987039\\
-0.125146926879917	-0.811288703531837\\
-0.125957699304119	-0.810255555323797\\
-0.126767436812625	-0.809218873297301\\
-0.12757613587509	-0.808178664400929\\
-0.128383792968124	-0.807134935597417\\
-0.129190404575308	-0.806087693863616\\
-0.129995967187205	-0.80503694619045\\
-0.130800477301377	-0.803982699582872\\
-0.131603931422398	-0.802924961059828\\
-0.132406326061869	-0.801863737654205\\
-0.133207657738428	-0.8007990364128\\
-0.13400792297777	-0.799730864396267\\
-0.134807118312654	-0.798659228679084\\
-0.135605240282923	-0.797584136349505\\
-0.136402285435516	-0.79650559450952\\
-0.137198250324477	-0.795423610274812\\
-0.137993131510977	-0.794338190774716\\
-0.138786925563319	-0.793249343152172\\
-0.139579629056959	-0.79215707456369\\
-0.140371238574514	-0.791061392179299\\
-0.14116175070578	-0.789962303182512\\
-0.141951162047741	-0.788859814770277\\
-0.142739469204585	-0.78775393415294\\
-0.143526668787719	-0.786644668554199\\
-0.144312757415779	-0.785532025211059\\
-0.145097731714643	-0.784416011373795\\
-0.145881588317449	-0.783296634305904\\
-0.146664323864604	-0.782173901284066\\
-0.147445935003796	-0.781047819598098\\
-0.148226418390013	-0.779918396550913\\
-0.149005770685549	-0.778785639458473\\
-0.149783988560024	-0.777649555649754\\
-0.15056106869039	-0.776510152466693\\
-0.151337007760951	-0.775367437264153\\
-0.152111802463368	-0.774221417409875\\
-0.15288544949668	-0.773072100284437\\
-0.153657945567311	-0.771919493281209\\
-0.154429287389085	-0.77076360380631\\
-0.155199471683239	-0.769604439278567\\
-0.155968495178435	-0.768442007129468\\
-0.156736354610772	-0.767276314803121\\
-0.157503046723801	-0.76610736975621\\
-0.158268568268534	-0.76493517945795\\
-0.15903291600346	-0.763759751390045\\
-0.159796086694555	-0.762581093046645\\
-0.160558077115297	-0.761399211934299\\
-0.161318884046674	-0.760214115571915\\
-0.162078504277201	-0.759025811490715\\
-0.16283693460293	-0.75783430723419\\
-0.163594171827463	-0.756639610358057\\
-0.164350212761962	-0.755441728430217\\
-0.165105054225166	-0.754240669030705\\
-0.165858693043397	-0.753036439751656\\
-0.166611126050577	-0.751829048197251\\
-0.167362350088238	-0.75061850198368\\
-0.168112362005534	-0.749404808739091\\
-0.168861158659252	-0.748187976103557\\
-0.169608736913828	-0.746968011729018\\
-0.170355093641352	-0.745744923279247\\
-0.171100225721587	-0.744518718429804\\
-0.171844130041974	-0.743289404867988\\
-0.172586803497651	-0.742056990292794\\
-0.173328242991459	-0.740821482414872\\
-0.174068445433954	-0.73958288895648\\
-0.174807407743423	-0.738341217651439\\
-0.17554512684589	-0.73709647624509\\
-0.176281599675131	-0.735848672494249\\
-0.177016823172686	-0.734597814167161\\
-0.177750794287866	-0.733343909043458\\
-0.178483509977771	-0.732086964914116\\
-0.179214967207293	-0.730826989581402\\
-0.179945162949137	-0.729563990858839\\
-0.180674094183823	-0.728297976571158\\
-0.181401757899703	-0.72702895455425\\
-0.18212815109297	-0.725756932655125\\
-0.182853270767671	-0.724481918731867\\
-0.183577113935716	-0.723203920653588\\
-0.184299677616887	-0.721922946300381\\
-0.185020958838855	-0.720639003563282\\
-0.185740954637186	-0.719352100344216\\
-0.186459662055353	-0.718062244555961\\
-0.187177078144749	-0.716769444122096\\
-0.187893199964694	-0.71547370697696\\
-0.188608024582448	-0.714175041065604\\
-0.189321549073221	-0.71287345434375\\
-0.190033770520187	-0.711568954777742\\
-0.190744686014487	-0.710261550344503\\
-0.191454292655249	-0.708951249031489\\
-0.19216258754959	-0.707638058836644\\
-0.192869567812631	-0.706321987768355\\
-0.193575230567508	-0.705003043845406\\
-0.19427957294538	-0.703681235096934\\
-0.19498259208544	-0.702356569562383\\
-0.195684285134926	-0.701029055291458\\
-0.196384649249131	-0.699698700344081\\
-0.197083681591412	-0.698365512790344\\
-0.197781379333203	-0.697029500710466\\
-0.198477739654021	-0.695690672194743\\
-0.19917275974148	-0.69434903534351\\
-0.199866436791299	-0.693004598267086\\
-0.200558768007312	-0.691657369085737\\
-0.201249750601479	-0.690307355929626\\
-0.201939381793893	-0.688954566938769\\
-0.202627658812795	-0.687599010262986\\
-0.203314578894578	-0.686240694061861\\
-0.2040001392838	-0.684879626504692\\
-0.204684337233194	-0.683515815770448\\
-0.205367170003678	-0.68214927004772\\
-0.20604863486436	-0.680779997534679\\
-0.206728729092553	-0.679408006439027\\
-0.207407449973782	-0.678033304977954\\
-0.208084794801795	-0.67665590137809\\
-0.208760760878571	-0.675275803875461\\
-0.209435345514328	-0.67389302071544\\
-0.210108546027534	-0.672507560152707\\
-0.21078035974492	-0.671119430451196\\
-0.211450784001481	-0.669728639884053\\
-0.212119816140493	-0.668335196733591\\
-0.212787453513518	-0.666939109291241\\
-0.213453693480412	-0.665540385857507\\
-0.214118533409339	-0.664139034741923\\
-0.214781970676775	-0.662735064263002\\
-0.215444002667519	-0.661328482748193\\
-0.216104626774704	-0.659919298533835\\
-0.216763840399802	-0.658507519965109\\
-0.217421640952634	-0.657093155395993\\
-0.218078025851381	-0.655676213189217\\
-0.21873299252259	-0.654256701716214\\
-0.219386538401185	-0.652834629357077\\
-0.220038660930472	-0.65141000450051\\
-0.220689357562152	-0.649982835543785\\
-0.221338625756328	-0.648553130892691\\
-0.221986462981511	-0.647120898961491\\
-0.222632866714632	-0.645686148172878\\
-0.223277834441048	-0.644248886957921\\
-0.223921363654552	-0.642809123756029\\
-0.22456345185738	-0.641366867014895\\
-0.225204096560221	-0.639922125190454\\
-0.225843295282221	-0.63847490674684\\
-0.226481045550999	-0.637025220156333\\
-0.227117344902645	-0.635573073899314\\
-0.227752190881738	-0.634118476464225\\
-0.228385581041346	-0.632661436347512\\
-0.22901751294304	-0.63120196205359\\
-0.229647984156898	-0.629740062094786\\
-0.230276992261515	-0.6282757449913\\
-0.230904534844009	-0.626809019271153\\
-0.23153060950003	-0.625339893470147\\
-0.23215521383377	-0.62386837613181\\
-0.232778345457965	-0.622394475807358\\
-0.233400001993908	-0.620918201055642\\
-0.234020181071456	-0.619439560443105\\
-0.234638880329032	-0.617958562543733\\
-0.235256097413642	-0.61647521593901\\
-0.235871829980872	-0.614989529217873\\
-0.236486075694904	-0.61350151097666\\
-0.237098832228519	-0.612011169819067\\
-0.237710097263107	-0.610518514356105\\
-0.238319868488669	-0.609023553206044\\
-0.238928143603831	-0.607526294994374\\
-0.239534920315848	-0.606026748353755\\
-0.240140196340609	-0.604524921923972\\
-0.240743969402649	-0.603020824351887\\
-0.24134623723515	-0.601514464291392\\
-0.241946997579955	-0.600005850403362\\
-0.24254624818757	-0.598494991355612\\
-0.243143986817172	-0.596981895822842\\
-0.243740211236615	-0.595466572486601\\
-0.24433491922244	-0.593949030035232\\
-0.244928108559878	-0.592429277163826\\
-0.245519777042861	-0.59090732257418\\
-0.246109922474023	-0.589383174974745\\
-0.246698542664712	-0.587856843080582\\
-0.247285635434994	-0.586328335613314\\
-0.247871198613657	-0.584797661301079\\
-0.248455230038225	-0.583264828878486\\
-0.249037727554958	-0.581729847086561\\
-0.249618689018858	-0.580192724672709\\
-0.25019811229368	-0.57865347039066\\
-0.250775995251936	-0.577112093000427\\
-0.251352335774901	-0.575568601268256\\
-0.251927131752616	-0.574023003966581\\
-0.252500381083903	-0.572475309873974\\
-0.253072081676362	-0.570925527775101\\
-0.253642231446381	-0.569373666460675\\
-0.254210828319144	-0.567819734727409\\
-0.254777870228631	-0.566263741377965\\
-0.255343355117631	-0.564705695220914\\
-0.255907280937742	-0.563145605070682\\
-0.256469645649381	-0.561583479747509\\
-0.257030447221788	-0.560019328077398\\
-0.25758968363303	-0.558453158892069\\
-0.258147352870012	-0.556884981028912\\
-0.258703452928476	-0.555314803330944\\
-0.259257981813012	-0.553742634646752\\
-0.259810937537059	-0.552168483830457\\
-0.260362318122916	-0.55059235974166\\
-0.26091212160174	-0.549014271245399\\
-0.26146034601356	-0.547434227212099\\
-0.262006989407277	-0.545852236517526\\
-0.262552049840669	-0.54426830804274\\
-0.263095525380398	-0.542682450674049\\
-0.263637414102016	-0.541094673302959\\
-0.264177714089969	-0.539504984826133\\
-0.2647164234376	-0.537913394145336\\
-0.26525354024716	-0.536319910167392\\
-0.265789062629807	-0.53472454180414\\
-0.266322988705612	-0.53312729797238\\
-0.266855316603569	-0.531528187593833\\
-0.267386044461593	-0.529927219595088\\
-0.26791517042653	-0.528324402907558\\
-0.268442692654157	-0.526719746467436\\
-0.268968609309194	-0.52511325921564\\
-0.2694929185653	-0.523504950097772\\
-0.270015618605084	-0.521894828064069\\
-0.270536707620108	-0.520282902069359\\
-0.271056183810889	-0.518669181073007\\
-0.271574045386908	-0.517053674038875\\
-0.27209029056661	-0.515436389935271\\
-0.272604917577412	-0.513817337734905\\
-0.273117924655707	-0.512196526414837\\
-0.273629310046863	-0.510573964956435\\
-0.274139072005235	-0.508949662345326\\
-0.274647208794166	-0.507323627571348\\
-0.275153718685988	-0.505695869628503\\
-0.275658599962032	-0.504066397514914\\
-0.276161850912629	-0.502435220232773\\
-0.27666346983711	-0.500802346788293\\
-0.277163455043821	-0.499167786191669\\
-0.277661804850114	-0.497531547457021\\
-0.278158517582361	-0.495893639602354\\
-0.278653591575953	-0.49425407164951\\
-0.279147025175303	-0.492612852624115\\
-0.279638816733853	-0.490969991555541\\
-0.280128964614077	-0.489325497476854\\
-0.280617467187483	-0.487679379424764\\
-0.281104322834616	-0.486031646439587\\
-0.281589529945065	-0.484382307565189\\
-0.282073086917466	-0.482731371848942\\
-0.2825549921595	-0.481078848341681\\
-0.283035244087903	-0.479424746097652\\
-0.283513841128468	-0.477769074174465\\
-0.283990781716046	-0.476111841633051\\
-0.28446606429455	-0.474453057537613\\
-0.284939687316959	-0.472792730955577\\
-0.285411649245322	-0.471130870957548\\
-0.28588194855076	-0.469467486617261\\
-0.286350583713468	-0.467802587011539\\
-0.286817553222721	-0.466136181220236\\
-0.287282855576875	-0.464468278326201\\
-0.287746489283368	-0.462798887415224\\
-0.288208452858729	-0.461128017575992\\
-0.288668744828574	-0.459455677900043\\
-0.289127363727615	-0.457781877481715\\
-0.289584308099656	-0.456106625418105\\
-0.290039576497602	-0.454429930809016\\
-0.29049316748346	-0.452751802756914\\
-0.290945079628337	-0.451072250366883\\
-0.29139531151245	-0.44939128274657\\
-0.291843861725123	-0.447708909006151\\
-0.292290728864793	-0.44602513825827\\
-0.292735911539009	-0.444339979618003\\
-0.293179408364437	-0.442653442202806\\
-0.293621217966863	-0.440965535132471\\
-0.294061338981191	-0.439276267529075\\
-0.294499770051451	-0.437585648516938\\
-0.294936509830798	-0.435893687222576\\
-0.295371556981512	-0.434200392774648\\
-0.295804910175006	-0.432505774303919\\
-0.296236568091823	-0.430809840943204\\
-0.296666529421641	-0.429112601827329\\
-0.297094792863272	-0.427414066093078\\
-0.297521357124667	-0.425714242879151\\
-0.297946220922917	-0.424013141326114\\
-0.298369382984254	-0.422310770576357\\
-0.298790842044052	-0.420607139774041\\
-0.299210596846833	-0.418902258065056\\
-0.299628646146263	-0.417196134596973\\
-0.300044988705157	-0.415488778518999\\
-0.300459623295481	-0.413780198981927\\
-0.300872548698352	-0.412070405138093\\
-0.301283763704039	-0.410359406141327\\
-0.301693267111968	-0.408647211146908\\
-0.302101057730719	-0.406933829311516\\
-0.30250713437803	-0.405219269793189\\
-0.302911495880798	-0.403503541751272\\
-0.303314141075079	-0.401786654346372\\
-0.30371506880609	-0.400068616740314\\
-0.304114277928213	-0.398349438096092\\
-0.304511767304991	-0.396629127577824\\
-0.304907535809133	-0.394907694350704\\
-0.305301582322512	-0.393185147580958\\
-0.305693905736169	-0.391461496435794\\
-0.306084504950314	-0.389736750083361\\
-0.306473378874323	-0.388010917692698\\
-0.306860526426743	-0.38628400843369\\
-0.307245946535291	-0.38455603147702\\
-0.307629638136855	-0.382826995994126\\
-0.308011600177495	-0.381096911157152\\
-0.308391831612443	-0.379365786138901\\
-0.308770331406103	-0.377633630112793\\
-0.309147098532057	-0.375900452252813\\
-0.309522131973056	-0.374166261733471\\
-0.309895430721029	-0.372431067729752\\
-0.310266993777079	-0.37069487941707\\
-0.310636820151485	-0.368957705971223\\
-0.311004908863703	-0.367219556568347\\
-0.311371258942362	-0.365480440384871\\
-0.311735869425271	-0.363740366597466\\
-0.312098739359414	-0.361999344383007\\
-0.312459867800954	-0.360257382918519\\
-0.312819253815227	-0.358514491381137\\
-0.313176896476751	-0.356770678948056\\
-0.313532794869217	-0.35502595479649\\
-0.313886948085496	-0.353280328103619\\
-0.314239355227636	-0.351533808046549\\
-0.314590015406861	-0.349786403802266\\
-0.314938927743571	-0.348038124547585\\
-0.315286091367345	-0.34628897945911\\
-0.315631505416937	-0.344538977713186\\
-0.315975169040277	-0.342788128485852\\
-0.316317081394473	-0.341036440952798\\
-0.316657241645806	-0.339283924289315\\
-0.316995648969733	-0.337530587670255\\
-0.317332302550886	-0.335776440269983\\
-0.317667201583071	-0.334021491262327\\
-0.318000345269266	-0.332265749820539\\
-0.318331732821625	-0.330509225117248\\
-0.318661363461471	-0.328751926324409\\
-0.318989236419301	-0.326993862613265\\
-0.319315350934781	-0.325235043154298\\
-0.31963970625675	-0.323475477117182\\
-0.319962301643213	-0.321715173670741\\
-0.320283136361344	-0.319954141982901\\
-0.320602209687487	-0.318192391220645\\
-0.320919520907149	-0.316429930549968\\
-0.321235069315006	-0.314666769135834\\
-0.321548854214894	-0.312902916142126\\
-0.321860874919818	-0.311138380731604\\
-0.322171130751939	-0.30937317206586\\
-0.322479621042584	-0.307607299305271\\
-0.322786345132238	-0.305840771608954\\
-0.323091302370543	-0.304073598134723\\
-0.3233944921163	-0.302305788039039\\
-0.323695913737466	-0.300537350476973\\
-0.32399556661115	-0.298768294602152\\
-0.324293450123616	-0.296998629566722\\
-0.32458956367028	-0.295228364521295\\
-0.324883906655705	-0.293457508614911\\
-0.325176478493605	-0.291686070994988\\
-0.325467278606839	-0.289914060807282\\
-0.325756306427412	-0.288141487195838\\
-0.326043561396471	-0.286368359302946\\
-0.326329042964304	-0.284594686269096\\
-0.326612750590341	-0.282820477232937\\
-0.326894683743148	-0.281045741331227\\
-0.327174841900427	-0.279270487698789\\
-0.327453224549013	-0.277494725468472\\
-0.327729831184875	-0.275718463771097\\
-0.328004661313109	-0.273941711735421\\
-0.328277714447942	-0.272164478488089\\
-0.328548990112723	-0.270386773153586\\
-0.328818487839928	-0.268608604854199\\
-0.32908620717115	-0.266829982709967\\
-0.329352147657105	-0.265050915838641\\
-0.329616308857624	-0.263271413355635\\
-0.32987869034165	-0.261491484373987\\
-0.330139291687242	-0.259711138004307\\
-0.330398112481566	-0.257930383354743\\
-0.330655152320894	-0.256149229530925\\
-0.330910410810604	-0.254367685635931\\
-0.331163887565175	-0.252585760770237\\
-0.331415582208187	-0.250803464031674\\
-0.331665494372313	-0.249020804515385\\
-0.331913623699322	-0.247237791313779\\
-0.332159969840075	-0.245454433516488\\
-0.332404532454519	-0.243670740210324\\
-0.332647311211688	-0.241886720479233\\
-0.332888305789696	-0.240102383404251\\
-0.333127515875741	-0.238317738063463\\
-0.333364941166093	-0.236532793531955\\
-0.333600581366098	-0.234747558881775\\
-0.333834436190173	-0.232962043181881\\
-0.334066505361801	-0.231176255498109\\
-0.334296788613529	-0.229390204893119\\
-0.334525285686967	-0.227603900426356\\
-0.33475199633278	-0.225817351154004\\
-0.33497692031069	-0.224030566128946\\
-0.33520005738947	-0.222243554400718\\
-0.33542140734694	-0.220456325015463\\
-0.335640969969963	-0.218668887015893\\
-0.335858745054447	-0.216881249441241\\
-0.336074732405332	-0.21509342132722\\
-0.336288931836598	-0.213305411705979\\
-0.336501343171252	-0.211517229606056\\
-0.336711966241326	-0.209728884052344\\
-0.336920800887879	-0.207940384066036\\
-0.337127846960986	-0.206151738664591\\
-0.337333104319741	-0.204362956861686\\
-0.337536572832246	-0.202574047667175\\
-0.337738252375613	-0.200785020087043\\
-0.337938142835959	-0.198995883123368\\
-0.338136244108398	-0.197206645774271\\
-0.338332556097044	-0.195417317033881\\
-0.338527078714999	-0.193627905892285\\
-0.338719811884357	-0.19183842133549\\
-0.338910755536192	-0.190048872345375\\
-0.339099909610562	-0.188259267899655\\
-0.339287274056498	-0.186469616971831\\
-0.339472848832002	-0.184679928531155\\
-0.339656633904044	-0.182890211542577\\
-0.339838629248558	-0.181100474966716\\
-0.340018834850434	-0.179310727759802\\
-0.340197250703518	-0.177520978873647\\
-0.340373876810604	-0.175731237255592\\
-0.340548713183432	-0.173941511848474\\
-0.340721759842684	-0.172151811590575\\
-0.340893016817975	-0.170362145415583\\
-0.341062484147852	-0.168572522252553\\
-0.341230161879792	-0.16678295102586\\
-0.34139605007019	-0.164993440655156\\
-0.341560148784359	-0.163204000055335\\
-0.341722458096527	-0.161414638136481\\
-0.341882978089828	-0.159625363803834\\
-0.342041708856297	-0.157836185957743\\
-0.34219865049687	-0.156047113493627\\
-0.342353803121375	-0.154258155301929\\
-0.342507166848527	-0.152469320268081\\
-0.342658741805924	-0.150680617272453\\
-0.342808528130043	-0.14889205519032\\
-0.342956525966233	-0.147103642891812\\
-0.34310273546871	-0.14531538924188\\
-0.343247156800553	-0.14352730310025\\
-0.343389790133698	-0.141739393321379\\
-0.343530635648933	-0.13995166875442\\
-0.343669693535892	-0.138164138243174\\
-0.343806963993051	-0.136376810626052\\
-0.34394244722772	-0.134589694736034\\
-0.344076143456041	-0.132802799400624\\
-0.34420805290298	-0.131016133441813\\
-0.344338175802323	-0.129229705676033\\
-0.344466512396667	-0.127443524914122\\
-0.34459306293742	-0.125657599961274\\
-0.344717827684792	-0.123871939617008\\
-0.344840806907789	-0.122086552675117\\
-0.344962000884206	-0.120301447923634\\
-0.345081409900627	-0.118516634144789\\
-0.345199034252412	-0.116732120114967\\
-0.345314874243697	-0.114947914604665\\
-0.345428930187383	-0.113164026378458\\
-0.345541202405134	-0.111380464194949\\
-0.345651691227371	-0.109597236806738\\
-0.345760396993262	-0.107814352960373\\
-0.34586732005072	-0.106031821396313\\
-0.345972460756394	-0.104249650848887\\
-0.346075819475666	-0.102467850046253\\
-0.346177396582642	-0.10068642771036\\
-0.346277192460147	-0.0989053925569027\\
-0.34637520749972	-0.0971247532952835\\
-0.346471442101603	-0.0953445186285744\\
-0.346565896674741	-0.093564697253472\\
-0.34665857163677	-0.091785297860262\\
-0.346749467414016	-0.0900063291327747\\
-0.346838584441484	-0.0882277997483486\\
-0.346925923162851	-0.0864497183777868\\
-0.347011484030465	-0.0846720936853192\\
-0.347095267505334	-0.0828949343285623\\
-0.347177274057118	-0.0811182489584789\\
-0.347257504164128	-0.0793420462193359\\
-0.347335958313313	-0.07756633474867\\
-0.347412637000258	-0.0757911231772416\\
-0.347487540729175	-0.0740164201290002\\
-0.347560670012897	-0.07224223422104\\
-0.347632025372868	-0.0704685740635651\\
-0.347701607339142	-0.0686954482598454\\
-0.347769416450371	-0.0669228654061813\\
-0.3478354532538	-0.065150834091859\\
-0.347899718305262	-0.0633793628991177\\
-0.347962212169165	-0.0616084604031042\\
-0.34802293541849	-0.059838135171837\\
-0.348081888634784	-0.0580683957661655\\
-0.348139072408149	-0.0562992507397318\\
-0.348194487337238	-0.0545307086389309\\
-0.348248134029248	-0.0527627780028719\\
-0.348300013099909	-0.0509954673633386\\
-0.348350125173482	-0.0492287852447517\\
-0.348398470882745	-0.0474627401641273\\
-0.348445050868993	-0.0456973406310409\\
-0.348489865782024	-0.0439325951475869\\
-0.348532916280136	-0.0421685122083411\\
-0.348574203030117	-0.0404051003003194\\
-0.34861372670724	-0.0386423679029432\\
-0.348651487995251	-0.036880323487996\\
-0.348687487586365	-0.0351189755195901\\
-0.348721726181258	-0.0333583324541229\\
-0.348754204489057	-0.0315984027402435\\
-0.348784923227336	-0.0298391948188097\\
-0.348813883122103	-0.0280807171228534\\
-0.348841084907799	-0.0263229780775397\\
-0.348866529327281	-0.0245659861001308\\
-0.348890217131825	-0.0228097495999458\\
-0.348912149081107	-0.0210542769783253\\
-0.348932325943206	-0.0192995766285902\\
-0.348950748494584	-0.0175456569360069\\
-0.348967417520089	-0.0157925262777463\\
-0.348982333812941	-0.0140401930228489\\
-0.348995498174723	-0.0122886655321851\\
-0.349006911415377	-0.0105379521584193\\
-0.349016574353194	-0.00878806124596787\\
-0.349024487814802	-0.0070390011309693\\
-0.349030652635164	-0.00529078014123736\\
-0.349035069657566	-0.00354340659623242\\
-0.349037739733609	-0.00179688880701721\\
-0.3490386637232	-5.12350762239489e-05\\
}
[postaction={decorate, decoration={markings,
        mark=between positions 0.5 and 1 step 1 with {\arrow[blue,line width=1.5pt]{latex};}
      }}]
;
%\addlegendentry{data7}

\addplot [color=blue, dotted, line width=2.0pt]
  table[row sep=crcr]{%
-0.3490386637232	-5.12350762239489e-05\\
-0.3490386637232	-5.12350762239489e-05\\
};
%\addlegendentry{data8}

\addplot [color=black, line width=2.0pt]
  table[row sep=crcr]{%
-0.3490386637232	-5.12350762239489e-05\\
-0.349073845291885	-0.0049025415292583\\
-0.349174834839881	-0.00942392008489726\\
-0.349337160237551	-0.0136369045939839\\
-0.349556618031017	-0.0175617308633371\\
-0.350589098065917	-0.0283693086439765\\
-0.352054991307645	-0.0369568414065178\\
-0.353869301471628	-0.0437323703444827\\
-0.355953880196747	-0.0490685424552535\\
-0.358276960488466	-0.0533159170008453\\
-0.360770421997525	-0.0565933138036604\\
-0.363396637027162	-0.0590771979020226\\
-0.366120854121684	-0.0609291306509354\\
-0.369454315353968	-0.0624920767918119\\
-0.372856596864769	-0.0634788134679982\\
-0.376302373825639	-0.0640018964571114\\
-0.3797667873332	-0.0641708331722673\\
-0.383840042592252	-0.064033388171067\\
-0.38789497543965	-0.0635909162128133\\
-0.391916865064456	-0.0629014529014435\\
-0.395889351787384	-0.062030762484929\\
-0.400554785011872	-0.0608258624484678\\
-0.405122774133902	-0.0594593478628749\\
-0.409585426815109	-0.0579563238661469\\
-0.413930853742804	-0.0563618436012937\\
-0.418883440040233	-0.0544217977991724\\
-0.42365936077026	-0.0524035452504142\\
-0.428256838266265	-0.0503133741971886\\
-0.432666831850687	-0.0481937432344772\\
-0.43737416413842	-0.0458375824651464\\
-0.441845404340102	-0.043471307759932\\
-0.446086043860177	-0.0410910821275353\\
-0.450091004318601	-0.0387436314277592\\
-0.454112231467813	-0.0363209986176885\\
-0.457876954565063	-0.0339505776791108\\
-0.461397461200061	-0.0316228970163229\\
-0.464674249676417	-0.029381693778642\\
-0.46789688354838	-0.0271383732372085\\
-0.47086955986795	-0.0249948092781448\\
-0.473609803532583	-0.0229362897973825\\
-0.476123174455512	-0.0209986150182565\\
-0.478589846718732	-0.0190800882282073\\
-0.480828290630336	-0.0172891376986153\\
-0.482859639977225	-0.0156054605384096\\
-0.484692963317551	-0.0140565026540394\\
-0.486358199893207	-0.0126473754471209\\
-0.487854972913385	-0.0113548970347892\\
-0.489200557820782	-0.0101626884743704\\
-0.49040488606825	-0.00908139523460288\\
-0.491477752432714	-0.00811843330301848\\
-0.49243614251785	-0.00724637835553415\\
-0.493292388615766	-0.00645355298160592\\
-0.494054986473921	-0.00574117214346382\\
-0.494731958781627	-0.00510915756213757\\
-0.495334093124502	-0.00454195160644937\\
-0.495869697896324	-0.00403159370053602\\
-0.496345138437993	-0.00357595554938436\\
-0.496495003667647	-0.00343247619491019\\
-0.496638850141129	-0.00329449464936615\\
-0.496776909292576	-0.00316182152339189\\
-0.496909404014354	-0.00303427532018085\\
}
[postaction={decorate, decoration={markings,
        mark=between positions 0.7 and 1 step 1 with {\arrow[black,line width=1.5pt]{latex};}
      }}]
;
%\addlegendentry{data9}

\addplot [color=black, line width=1.0pt, draw=none, mark=*, mark options={solid, fill=gray, black}]
  table[row sep=crcr]{%
-0.50083349533071	0.000837868594377482\\
-0.50083349533071	0.000855510893233957\\
};
%\addlegendentry{data10}

\addplot [color=black, line width=1.0pt, draw=none, mark=*, mark options={solid, fill=gray, black}]
  table[row sep=crcr]{%
0.497022788413045	0.00292482868251474\\
0.497022788413045	0.002988519060116\\
};
%\addlegendentry{data11}

\end{axis}
\end{tikzpicture}%