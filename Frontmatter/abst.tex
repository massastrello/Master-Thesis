\chapter*{Abstract}
\label{abst}

%%%%%%%%%%%%%%%%%%%%%%%%%%%%%%%%%%%%%%%%%%%%%%%%%%%%%%%%%%%%%%%%%%%%%%%%%%%%%%%
Accurate and robust control of robots in highly dynamic tasks is arguably one of the hardest open problems in robotics. The incapability of robots to safely and reliably interact with their environment is what still refrain them from becoming ubiquitous in our society.
%
The main challenge is due to the \textit{nonsmooth} nature of such dynamic tasks, as they often involve impacts between parts of the robot and its environment. Examples are \textit{legged locomotion}, \textit{non--prehensile} manipulation or aerial robot landing.
%
\newline

%
This issue in robotics is a fundamental control challenge, extremely appealing for the research community, which is trying to fulfill the needs of industries who are seeking more autonomy for commercial robots.
This interest expands swiftly to a theoretical level once the problem is formalized. Discontinuities of any sort used to be sworn enemies of most control theorists. However, in recent years, a new emerging field of control theory: the \textit{hybrid dynamical systems}, promises a complete set of mathematical tools to deal with tasks characterized by interacting continuous and discrete time dynamics. Besides, this framework is by no means close to solve the robotics problem. Control theorists often spend too much time on over--complicated math without foresight of its applications. 
\newline

%
Nevertheless, the work presented in this thesis hinges on the most fundamental concept of classical and modern physics: \textit{energy}. Energy is the key to understand, and thus control, the behavior of dynamical system by physical insights. The most relevant mathematical tool in this context is the one of \textit{port--Hamiltonian systems}. The aim of this thesis is to provide a unified modeling framework merging the ones of port--Hamiltonian and hybrid systems through consistent and practically useful results.
%
\newline

%
The research work embraces four different objectives: 
characterize a new unified modeling strategy to tackle highly dynamic robotic tasks; apply the theory to cope with the challenging \textit{ball--dribbling robot} problem; show the broad applications' spectrum of the developed framework by adopting it into to bear pure control theory problem; perform a first step towards real implementation of such control systems through system identification task.
%
\newline

%
The contribution of this thesis is primarily theoretical, however, all chapters are application--oriented displaying real examples with simulations.


%Accurate localization and mapping are very important for mobile robots working in indoor environments where human exists. In order to localize and map accurately, a rangefinder (LRF) is widely used because of it high accuracy in distance measuring. In modern indoor environments, glass is very common. However, glass is usually not shown on the map correctly when using LRF, because it cannot be detected by LRF from all angles, as other objects. The incorrect map brings the danger of crashing into the glass to the mobile robot when path planning. Furthermore, the limited ability of LRF in glass detection negatively influences the localization accuracy of the mobile robots. 
%
%To solve the problems mentioned above, this thesis proposes to build a glass confidence map which can show all objects on the map, glass included, and their probability of being glass. The purpose of this thesis is to build such a glass confidence map. The method proposed is to first classify glass and non-glass objects, and then process them differently in order to get a better map. 
%
%%The reason glass need to be processed differently is because in standard mapping stage, objects are assumed to be detectable from all incident angles, while glass does not match this assumption. 
%
%To classify the glass and non-glass objects, a neural network based classifier is proposed, with LRF's measured intensities, distances and incident angles as inputs, as well as glass probability as output. The classification method is based on the assumption proposed in this thesis, that the LRF's measured intensity is mainly influenced by material features, incident angle and distance, and therefore material features can be inferred by LRF intensity, distance and incident angle. To verified the assumption, experiments are performed, and the results show glass and non-glass panels have obviously different patterns and are separable in the feature space of intensity, distance and incident angle. The proposed neural network classifier has 2 hidden layers, 10 nodes in each hidden layer, and is trained and tested using experimental data. 
%
%Additionally, a novel map building method is proposed to build the targeted glass confidence map using the glass probability and information from standard Simultaneous Localization and Mapping (SLAM). First, the glass probability is registered to a temporary map, using a Gaussian Filter to reduce the influence of uncertainty in robot pose and LRF measurements. Second, to remove noise, the temporary map is filtered based on both occupancy probability and glass probability. Glass objects are applied to a lower occupancy threshold than non-glass objects, considering the fact glass cannot be detected consistently by the LRF.
%
%To verify the proposed method, two experiments in two different environments are performed. As a result, glass confidence maps are built, with more glass shown correctly than maps built by standard methods. Besides, quantitative analysis on the results shows that the neural network classifier classify glass and non-glass objects with high accuracy. In summary, the proposed method can build glass confidence maps successfully and accurately.

\clearpage
%%%%%%%%%%%%%%%%%%%%%%%%%%%%%%%%%%%%%%%%%%%%%%%%%%%%%%%%%%%%%%%%%%%%%%%%%%%%%%%
%%% Local Variables:
%%% mode: katex
%%% TeX-master: "../thesis"
%%% End:
